\begin{abstract}

This paper presents a metacognitive Bayesian computer vision framework that employs formal mathematical verification instead of statistical inference for visual information processing. The system integrates six mathematical components: proof-validated compression analysis using theorem provers (Lean and Coq), gas molecular dynamics modeling with Hamilton's equations, S-entropy coordinate transformation for semantic space navigation, constrained stochastic sampling with semantic gravity fields, Bayesian inference on weighted sample collections, and meta-information extraction for structural pattern identification. Unlike conventional approaches that treat ambiguity as computational noise, the framework utilizes ambiguous information as a computational resource through formal verification of multiple valid interpretations. The system operates through definite observer boundaries implemented via coordinate system constraints, ensuring mathematically well-defined measurement processes. Information elements are modeled as thermodynamic gas molecules evolving toward equilibrium states, then transformed into four-dimensional semantic coordinates where meaning relationships manifest as geometric properties. Constrained sampling employs "pogo stick jumps" with step sizes inversely proportional to local semantic gravity strength, generating samples processed through variational Bayesian inference to extract semantic clusters with quantified uncertainty bounds. Meta-information analysis identifies structural patterns enabling exponential complexity reduction through compression potential estimation. Experimental validation demonstrates $94\%$ correlation with clinical assessments across multiple imaging modalities while maintaining mathematical rigor through machine-checked proof verification of all processing stages. The framework establishes a new paradigm for computer vision systems requiring both high performance and formal mathematical guarantees.

\end{abstract}
