% Figures for Helicopter Life Sciences Paper
% Using ACTUAL images and charts from the analysis results

% Introduction Section - Thermodynamic Framework Concept
\begin{figure}[H]
\centering
\includegraphics[width=0.9\textwidth]{images/thermodynamic-intepretation.pdf}
\caption{Thermodynamic computer vision framework showing the conceptual transformation from biological image pixels to information gas molecules with thermodynamic properties, establishing the foundation for molecular dynamics, entropy coordinates, and meta-information extraction.}
\label{fig:thermodynamic-framework}
\end{figure}

% Fluorescence Microscopy Analysis Results

% Figure 1: Fluorescence Analysis Overview
\begin{figure}[H]
\centering
\includegraphics[width=0.9\textwidth]{images/figure1_fluorescence_overview.png}
\caption{Comprehensive fluorescence microscopy analysis overview showing multi-channel processing capabilities, background subtraction methods, segmentation results, and quantitative measurements across different biological samples.}
\label{fig:fluorescence-overview}
\end{figure}

% Figure 2: Detailed Fluorescence Analysis
\begin{figure}[H]
\centering
\includegraphics[width=0.9\textwidth]{images/figure2_fluorescence_detailed.png}
\caption{Detailed fluorescence analysis results showing segmentation performance, intensity measurements, signal-to-noise ratio analysis, morphological characterization, and colocalization metrics with statistical validation.}
\label{fig:fluorescence-detailed}
\end{figure}

% Figure 3: Segmentation Pipeline Conceptual Framework
\begin{figure}[H]
\centering
\includegraphics[width=0.9\textwidth]{images/segmentation-pipeline.pdf}
\caption{Fluorescence segmentation pipeline conceptual framework illustrating adaptive thresholding, morphological refinement, watershed segmentation, and quality assessment procedures for accurate structure identification.}
\label{fig:segmentation-pipeline}
\end{figure}

% Figure 4: Statistical Analysis Results
\begin{figure}[H]
\centering
\includegraphics[width=0.9\textwidth]{images/figure4_statistical_analysis.png}
\caption{Statistical analysis of fluorescence microscopy results showing distribution analysis, correlation studies, performance metrics, and comparative assessment across different imaging conditions and biological samples.}
\label{fig:statistical-analysis}
\end{figure}

% Figure 5: Time-Series Analysis
\begin{figure}[H]
\centering
\includegraphics[width=0.9\textwidth]{images/figure5_timeseries_detailed.png}
\caption{Time-series fluorescence analysis showing temporal dynamics, photobleaching correction, FRAP recovery analysis, kinetic parameter extraction, and dynamic process classification with quantitative metrics.}
\label{fig:timeseries-analysis}
\end{figure}

% Figure 6: Comparative Summary Analysis
\begin{figure}[H]
\centering
\includegraphics[width=0.9\textwidth]{images/figure6_fluorescence_comparative_summary.png}
\caption{Comparative fluorescence analysis summary showing performance across different sample types, method comparison, accuracy assessment, and integrated analysis results demonstrating framework effectiveness.}
\label{fig:fluorescence-comparative}
\end{figure}

% Multi-Channel Fluorescence Results
\begin{figure}[H]
\centering
\begin{subfigure}[t]{0.48\textwidth}
\includegraphics[width=\textwidth]{images/fluorescence_multi_channel.png}
\caption{Multi-channel fluorescence overlay}
\end{subfigure}
\hfill
\begin{subfigure}[t]{0.48\textwidth}
\includegraphics[width=\textwidth]{images/fluorescence_cellular_structure.png}
\caption{Cellular structure analysis}
\end{subfigure}
\caption{Multi-channel fluorescence microscopy results showing channel overlay analysis and cellular structure characterization with quantitative colocalization metrics.}
\label{fig:multichannel-fluorescence}
\end{figure}

% Comprehensive Fluorescence Analysis Results
\begin{figure}[H]
\centering
\begin{subfigure}[t]{0.32\textwidth}
\includegraphics[width=\textwidth]{images/fluorescence_comprehensive_cellular_structure_rfp.png}
\caption{RFP cellular analysis}
\end{subfigure}
\hfill
\begin{subfigure}[t]{0.32\textwidth}
\includegraphics[width=\textwidth]{images/fluorescence_comprehensive_fluorescence_cell_gfp.png}
\caption{GFP cell analysis}
\end{subfigure}
\hfill
\begin{subfigure}[t]{0.32\textwidth}
\includegraphics[width=\textwidth]{images/fluorescence_comprehensive_high_res_tissue_dapi.png}
\caption{DAPI tissue analysis}
\end{subfigure}
\caption{Comprehensive fluorescence analysis across different fluorophores showing RFP cellular structure analysis, GFP cell characterization, and DAPI tissue organization with integrated thermodynamic framework results.}
\label{fig:comprehensive-fluorescence}
\end{figure}

% Individual Sample Analysis Results
\begin{figure}[H]
\centering
\begin{subfigure}[t]{0.32\textwidth}
\includegraphics[width=\textwidth]{images/fluorescence_fluorescence_cell_gfp.png}
\caption{GFP fluorescence cell}
\end{subfigure}
\hfill
\begin{subfigure}[t]{0.32\textwidth}
\includegraphics[width=\textwidth]{images/fluorescence_high_res_tissue_dapi.png}
\caption{DAPI high-res tissue}
\end{subfigure}
\hfill
\begin{subfigure}[t]{0.32\textwidth}
\includegraphics[width=\textwidth]{images/fluorescence_cellular_structure_rfp.png}
\caption{RFP cellular structure}
\end{subfigure}
\caption{Individual fluorescence sample analysis results showing GFP cell characterization, DAPI tissue imaging, and RFP cellular structure analysis with quantitative metrics and thermodynamic integration.}
\label{fig:individual-fluorescence}
\end{figure}

% Electron Microscopy Analysis Framework

% Figure 7: Electron Microscopy Modality Framework
\begin{figure}[H]
\centering
\includegraphics[width=0.9\textwidth]{images/electron-microscope-modality.pdf}
\caption{Electron microscopy analysis framework showing TEM, SEM, and Cryo-EM modality-specific processing, parameter optimization, structure classification, and thermodynamic integration for ultrastructural characterization.}
\label{fig:em-modality-framework}
\end{figure}

% Electron Microscopy Analysis Results
\begin{figure}[H]
\centering
\begin{subfigure}[t]{0.32\textwidth}
\includegraphics[width=\textwidth]{images/electron_cellular_structure.png}
\caption{Cellular structure analysis}
\end{subfigure}
\hfill
\begin{subfigure}[t]{0.32\textwidth}
\includegraphics[width=\textwidth]{images/electron_fluorescence_cell.png}
\caption{Fluorescence cell EM}
\end{subfigure}
\hfill
\begin{subfigure}[t]{0.32\textwidth}
\includegraphics[width=\textwidth]{images/electron_high_res_tissue.png}
\caption{High-resolution tissue}
\end{subfigure}
\caption{Electron microscopy analysis results showing cellular structure characterization, fluorescence-correlated electron microscopy, and high-resolution tissue analysis with ultrastructural classification and morphological quantification.}
\label{fig:electron-microscopy-results}
\end{figure}

% Gas Molecular Dynamics Framework

% Figure 8: Intermolecular Forces Framework
\begin{figure}[H]
\centering
\includegraphics[width=0.9\textwidth]{images/intermolecular-potential-forces.pdf}
\caption{Gas molecular dynamics framework showing intermolecular potential forces, Lennard-Jones interactions, biological similarity factors, and equilibrium clustering behavior for thermodynamic image analysis.}
\label{fig:gas-molecular-forces}
\end{figure}

% Gas Molecular Dynamics Results
\begin{figure}[H]
\centering
\begin{subfigure}[t]{0.32\textwidth}
\includegraphics[width=\textwidth]{images/gas_molecular_cellular_structure.png}
\caption{Cellular structure dynamics}
\end{subfigure}
\hfill
\begin{subfigure}[t]{0.32\textwidth}
\includegraphics[width=\textwidth]{images/gas_molecular_fluorescence_cell.png}
\caption{Fluorescence cell dynamics}
\end{subfigure}
\hfill
\begin{subfigure}[t]{0.32\textwidth}
\includegraphics[width=\textwidth]{images/gas_molecular_high_res_tissue.png}
\caption{High-resolution tissue dynamics}
\end{subfigure}
\caption{Gas molecular dynamics analysis results showing cellular structure evolution, fluorescence cell molecular behavior, and high-resolution tissue dynamics with equilibrium clustering and biological interpretation.}
\label{fig:gas-molecular-results}
\end{figure}

% S-Entropy Coordinate System Framework

% Figure 9: S-Entropy Coordinate Framework
\begin{figure}[H]
\centering
\includegraphics[width=0.9\textwidth]{images/st-stellas-entropy.pdf}
\caption{S-Entropy coordinate system framework showing four-dimensional semantic transformation, structural complexity analysis, functional activity measurement, morphological diversity quantification, and temporal dynamics estimation.}
\label{fig:entropy-coordinate-framework}
\end{figure}

% S-Entropy Analysis Results
\begin{figure}[H]
\centering
\includegraphics[width=0.7\textwidth]{images/entropy_high_res_tissue.png}
\caption{S-Entropy coordinate analysis of high-resolution tissue showing four-dimensional coordinate transformation results, biological interpretation, trajectory analysis, and semantic space characterization with quantitative metrics.}
\label{fig:entropy-analysis-results}
\end{figure}

% Meta-Information Extraction Results
\begin{figure}[H]
\centering
\begin{subfigure}[t]{0.32\textwidth}
\includegraphics[width=\textwidth]{images/meta_cellular_structure.png}
\caption{Cellular structure meta-info}
\end{subfigure}
\hfill
\begin{subfigure}[t]{0.32\textwidth}
\includegraphics[width=\textwidth]{images/meta_fluorescence_cell.png}
\caption{Fluorescence cell meta-info}
\end{subfigure}
\hfill
\begin{subfigure}[t]{0.32\textwidth}
\includegraphics[width=\textwidth]{images/meta_high_res_tissue.png}
\caption{High-res tissue meta-info}
\end{subfigure}
\caption{Meta-information extraction results showing information type classification, semantic density analysis, compression potential assessment, and structural complexity quantification across different biological samples.}
\label{fig:meta-information-results}
\end{figure}

% Cross-Module Integration Framework

% Figure 10: Cross-Module Integration
\begin{figure}[H]
\centering
\includegraphics[width=0.9\textwidth]{images/cross-module-interaction.pdf}
\caption{Cross-module integration framework showing synergistic interactions between Gas Molecular Dynamics, S-Entropy Coordinate System, and Meta-Information Extraction modules with information flow, feedback mechanisms, and thermodynamic consistency.}
\label{fig:cross-module-integration}
\end{figure}

% Comprehensive Analysis Summary
\begin{figure}[H]
\centering
\includegraphics[width=0.9\textwidth]{images/figure_comparative_summary.png}
\caption{Comprehensive analysis summary showing integrated results across all modules, performance metrics, accuracy improvements through module integration, computational efficiency gains, and biological interpretation validation.}
\label{fig:comprehensive-summary}
\end{figure}

% Final Analysis Overview
\begin{figure}[H]
\centering
\includegraphics[width=0.9\textwidth]{images/figure1_fluorescence_analysis.png}
\caption{Integrated thermodynamic computer vision analysis showing the complete framework application to biological imaging, demonstrating the synergistic effects of molecular dynamics, entropy coordinates, and meta-information extraction for life sciences applications.}
\label{fig:integrated-analysis}
\end{figure}