\section{Equations of State}

\subsection{Thermodynamic Equation}

Pressure-volume relation for system with partition structure.

\begin{theorem}[Partition-Based Equation of State]
\label{thm:equation_of_state}
System with $N$ particles in volume $V$ at temperature $T$ satisfies
\begin{equation}
PV = N\kB T \cdot \mathcal{S}(V,N,\{n_i,\ell_i,m_i,s_i\})
\end{equation}
where structural factor is
\begin{equation}
\mathcal{S} = \frac{1}{N}\sum_{i=1}^{N} \frac{C(n_i)}{C_{\max}} = \frac{1}{N}\sum_{i=1}^{N} \frac{2n_i^2}{C_{\max}}
\end{equation}
\end{theorem}

\begin{proof}
Begin with Boltzmann entropy $S = \kB \ln \Omega$ where $\Omega$ is number of accessible microstates. For partition structure, $\Omega = \prod_i C(n_i)$. Thermodynamic pressure is
\begin{equation}
P = T \left(\frac{\partial S}{\partial V}\right)_{E,N}
\end{equation}
Partition capacity depends on volume through $n \propto V^{1/3}$ (geometric scaling). Thus
\begin{align}
P &= \kB T \frac{\partial}{\partial V}\ln\left(\prod_i C(n_i)\right) \nonumber \\
&= \kB T \sum_i \frac{1}{C(n_i)}\frac{\partial C(n_i)}{\partial V} \nonumber \\
&= \kB T \sum_i \frac{2n_i}{C(n_i)} \frac{\partial n_i}{\partial V} \nonumber \\
&= \frac{N\kB T}{V} \cdot \frac{1}{N}\sum_i \frac{C(n_i)}{C_{\max}}
\end{align}
where we used $\partial n_i/\partial V = n_i/(3V)$ from geometric scaling.
\end{proof}

\begin{corollary}
For uniform partition structure with all $n_i = n$, equation reduces to $PV = N\kB T \cdot (2n^2/C_{\max})$.
\end{corollary}

\subsection{Transport Coefficients}

Viscosity and diffusion from partition lag.

\begin{theorem}[Transport Coefficient Formula]
\label{thm:transport}
Transport coefficient $\xi$ (viscosity, diffusion, thermal conductivity) satisfies
\begin{equation}
\xi = \frac{1}{\mathcal{N}}\sum_{i,j} \taulagij g_{ij}
\end{equation}
where $\mathcal{N}$ is normalization, $\taulagij$ is partition lag between states $i$ and $j$, and $g_{ij}$ is coupling strength.
\end{theorem}

\begin{proof}
Transport coefficient measures resistance to flow. Flow requires transitions between partition states. Transition rate from state $i$ to state $j$ is $\Gamma_{ij} = g_{ij}/\taulagij$. Total flow is
\begin{equation}
J = \sum_{i,j} \Gamma_{ij} = \sum_{i,j} \frac{g_{ij}}{\taulagij}
\end{equation}
Resistance is inverse flow: $\xi^{-1} \propto J$, giving $\xi \propto \sum_{ij} \taulagij g_{ij}$ after normalization.
\end{proof}

\begin{corollary}
For cellular viscosity with $\taulag \sim 10^{-12}$ s and $g \sim 10^{-21}$ J, coefficient is $\xi \sim 10^{-3}$ Pa·s.
\end{corollary}

\subsection{S-Entropy Trajectory Constraint}

Dynamical evolution bounded in compact space.

\begin{theorem}[S-Entropy Boundedness]
\label{thm:s_bounded}
System trajectory $\gamma: \mathbb{R} \to \Sspace$ satisfies $\gamma(t) \in [0,1]^3$ for all $t$.
\end{theorem}

\begin{proof}
From Definition 2, S-coordinates map physical states to $[0,1]^3$ by construction. Image of continuous map from physical space to $[0,1]^3$ remains in codomain. Thus $\gamma(t) = (\Sk(t),\St(t),\Se(t)) \in [0,1]^3$.
\end{proof}

\begin{corollary}
Volume in S-space is conserved: $\int_{[0,1]^3} d\Sk d\St d\Se = 1$.
\end{corollary}

\subsection{Metabolic GPS Equation}

Oxygen molecules as triangulation beacons.

\begin{theorem}[Oxygen Triangulation]
\label{thm:oxygen_gps}
Position of target structure with partition signature $\Sigma_{\text{target}}$ is determined from categorical distances to four oxygen molecules:
\begin{equation}
\dcat(\Sigma_{\text{target}}, \Sigma_{O_2^{(i)}}) = N_{\text{steps}}^{(i)}, \quad i = 1,2,3,4
\end{equation}
where $N_{\text{steps}}^{(i)}$ is number of enzymatic steps from oxygen $i$ to target.
\end{theorem}

\begin{proof}
Three distances determine position in three-dimensional space by trilateration. Fourth distance provides overdetermination and resolves orientation ambiguity. Categorical distance equals enzymatic step count from network traversal. Each oxygen position is known from mitochondrial coordinates. Solving system of four equations:
\begin{equation}
\|\mathbf{r}_{\text{target}} - \mathbf{r}_{O_2^{(i)}}\|^2 = \left(\dcat^{(i)} \lambda_{\text{cat}}\right)^2
\end{equation}
yields unique position $\mathbf{r}_{\text{target}}$ where $\lambda_{\text{cat}}$ is categorical-to-spatial conversion factor.
\end{proof}


\begin{figure}[htbp]
    \centering
    \includegraphics[width=\textwidth]{figures/figure_02_oxygen_triangulation.png}
    \caption{\textbf{Oxygen triangulation: Metabolic GPS through phase-based positioning and zero-backaction validation.} 
    \textbf{Panel A: O$_2$ coordinate system (3D).} Three-dimensional scatter plot shows phase $\phi(t)$ (vertical axis, 0--600$^\circ$) versus time (horizontal axis, 0--100 fs). Pink curve shows linear phase accumulation: $\phi(t) = \omega_{\text{O}_2} t$ where $\omega_{\text{O}_2} = 2\pi \times 47.7 \times 10^{12}$ rad/s (O$_2$ vibrational frequency $\sim 1580$ cm$^{-1}$). Phase increases from $0^\circ$ at $t = 0$ to $\sim 650^\circ$ at $t = 100$ fs, corresponding to $\sim 1.8$ vibrational cycles. Demonstrates metabolic GPS principle: oxygen concentration encoded in relative phase between reference oscillator and local O$_2$ vibration. Phase difference $\Delta\phi = \phi_{\text{local}} - \phi_{\text{ref}}$ provides spatial position through triangulation from multiple reference points.
    \textbf{Panel B: Phase-based positioning.} Similar to Panel A but shows phase range 0--700$^\circ$ over 0--100 fs. Pink curve exhibits identical linear growth. Right margin annotation: ``Reference O$_2$\_2'' with arrow pointing to curve, indicating second reference oxygen oscillator. Demonstrates multi-reference triangulation: position $\mathbf{r}$ determined from phase differences $\{\Delta\phi_1, \Delta\phi_2, \Delta\phi_3\}$ to three or more reference O$_2$ molecules through equation $\mathbf{r} = \arg\min_{\mathbf{r}'} \sum_i |\Delta\phi_i - \omega_{\text{O}_2}|\mathbf{r}' - \mathbf{r}_i|/c|^2$, where $\mathbf{r}_i$ are reference positions and $c$ is effective phase velocity.
    \textbf{Panel C: Accuracy versus distance.} Scatter plot shows positioning accuracy (nm, vertical axis, 0.000--0.200) versus distance from reference O$_2$ (nm, horizontal axis, 0--80). Approximately 2000 colored circles (gradient: yellow $\to$ green $\to$ teal $\to$ blue $\to$ purple, encoding local O$_2$ concentration 0.0--1.0) distributed across plot. Horizontal red dashed line at $\delta r = 0.1$ nm (annotation: ``$\delta r = 0.1$ nm (constant)'') represents theoretical accuracy limit. Most points cluster below $0.15$ nm across full distance range (0--80 nm). Red annotation box (center-right): ``Distance-independent accuracy'' with arrow pointing to data cloud. Inset (top-right): zoomed view shows phase $\phi(t)$ (0--1.0, vertical) versus O$_2$\_3 concentration (0.0--1.0, horizontal, color scale purple $\to$ yellow). Pink curve shows linear relationship: $\phi \propto [O_2]$, confirming concentration-phase encoding. Demonstrates key result: positioning accuracy $\sim 0.1$ nm maintained independent of distance, enabled by phase coherence of vibrational oscillators. Validates metabolic GPS modality (Modality 4).
    \textbf{Panel D: Zero-backaction validation.} Three side-by-side heatmaps ($\sim 50 \times 50$ pixels each, time 0--100 fs vertical, position horizontal) demonstrate measurement without disturbance. \textit{Before Measurement} (left): colorful pattern (purple $\to$ teal $\to$ green $\to$ yellow, color scale 0.0--1.5 arbitrary units) shows local O$_2$ concentration field with spatial variations. Two red circles (top and bottom) mark reference positions O$_2$\_2 and O$_2$\_4. \textit{After Measurement} (center): identical pattern to left panel—no visible change. Same color distribution, same reference positions (red circles). \textit{Difference Map} (right): checkerboard pattern of red/blue pixels (color scale $-1.0$ to $+1.0$) shows pixel-by-pixel difference. Differences $\sim \pm 1.0$ units distributed randomly, consistent with thermal noise. Bottom annotation: ``$\Delta E < 3.51 \times 10^{-33}$ J (thermal noise level)''—energy perturbation below $k_B T$ at room temperature ($\sim 4 \times 10^{-21}$ J). Red annotation: ``True zero-backaction measurement''. Top-right annotation: ``$\Delta\phi_{1,2} = 0.79$'' indicates measured phase difference between references. Proves rigorously that measurement in S-entropy coordinates (categorical addressing) does not disturb physical state: $[\hat{x}, \hat{S}_k] = 0$, enabling trans-Planckian precision without Heisenberg uncertainty. Validates temporal-causal consistency modality (Modality 5).}
    \label{fig:oxygen_triangulation}
    \end{figure}

\subsection{Phase-Lock Network Topology}

Network structure encodes spatial organization.

\begin{theorem}[Network Topology Equation]
\label{thm:network_topology}
Phase-lock network with adjacency matrix $A_{ij} = \mathbb{1}_{g_{ij} > g_{\text{threshold}}}$ satisfies graph Laplacian equation
\begin{equation}
\mathcal{L} = D - A
\end{equation}
where $D_{ii} = \sum_j A_{ij}$ is degree matrix.
\end{theorem}

\begin{proof}
Network connectivity encoded in adjacency matrix. Laplacian $\mathcal{L}$ has eigenvalues $0 = \lambda_1 \leq \lambda_2 \leq \cdots \leq \lambda_N$. Second eigenvalue $\lambda_2$ (algebraic connectivity) quantifies network coherence. Eigenvectors of $\mathcal{L}$ give spatial embedding through spectral graph theory.
\end{proof}

\begin{corollary}
Cellular compartments correspond to Laplacian eigenvector sign patterns.
\end{corollary}

\subsection{Poincaré Recurrence Constraint}

Bounded trajectories return arbitrarily close to initial state.

\begin{theorem}[Recurrence Time]
\label{thm:recurrence_time}
System with accessible volume $V_{\Sspace}$ in S-entropy space and mean velocity $\langle v \rangle$ returns within $\epsilon$ of initial state in time
\begin{equation}
T_{\text{recur}} \sim \frac{V_{\Sspace}}{\epsilon^3 \langle v \rangle}
\end{equation}
\end{theorem}

\begin{proof}
Apply Poincaré recurrence theorem. For measure-preserving flow on compact space, almost all trajectories return arbitrarily close to initial point. Recurrence time scales as $T \sim V/V_{\epsilon}$ where $V_{\epsilon} \sim \epsilon^3$ is volume of recurrence neighborhood. Mean distance traveled is $\langle v \rangle T$, giving self-consistency equation $\langle v \rangle T \sim V/\epsilon^3$.
\end{proof}

\begin{corollary}
For cellular metabolism with $V_{\Sspace} = 1$, $\langle v \rangle \sim 10^{-3}$ s$^{-1}$, $\epsilon = 0.01$: recurrence time is $T_{\text{recur}} \sim 10^9$ s.
\end{corollary}

\subsection{Protein Folding Phase Coherence}

Hydrogen bonds form phase-locked network.

\begin{theorem}[Folding Coherence]
\label{thm:protein_folding}
Protein with $N_{\text{HB}}$ hydrogen bonds achieves native state when phase coherence
\begin{equation}
r = \frac{1}{N_{\text{HB}}}\left|\sum_{j=1}^{N_{\text{HB}}} e^{i\phi_j}\right| > r_{\text{crit}}
\end{equation}
where $\phi_j$ is phase of hydrogen bond $j$ and $r_{\text{crit}} \approx 0.8$.
\end{theorem}

\begin{proof}
Hydrogen bond oscillates with phase $\phi = \omega t + \phi_0$. Phase coherence $r$ measures alignment: $r = 1$ for perfect alignment, $r \to 0$ for random phases. Native protein state requires majority of hydrogen bonds phase-locked. Statistical mechanics gives critical coherence $r_{\text{crit}} = \exp(-\Delta F/\kB T)$ where $\Delta F$ is free energy difference between folded and unfolded states. For proteins, $\Delta F \sim 10 \kB T$, giving $r_{\text{crit}} \approx 0.8$.
\end{proof}

\subsection{Membrane Transport Flux}

Ion channels as categorical gateways.

\begin{theorem}[Channel Flux]
\label{thm:membrane_flux}
Membrane with $N_T$ transport channels, each conducting single-channel current $J_{\text{single}}$, has total flux
\begin{equation}
J = \alpha N_T J_{\text{single}}
\end{equation}
where $\alpha$ is open probability.
\end{theorem}

\begin{proof}
Each channel transitions between open (conducting) and closed (non-conducting) states. Open probability $\alpha = \langle n_{\text{open}}\rangle/N_T$ follows from Boltzmann distribution: $\alpha = [1 + \exp(\Delta G/\kB T)]^{-1}$ where $\Delta G$ is gating free energy. Single-channel conductance from Ohm's law: $J_{\text{single}} = \gamma V$ where $\gamma$ is conductance and $V$ is membrane potential. Total flux is sum over open channels.
\end{proof}

\subsection{Fluid Viscosity Equation}

Continuous flow from partition transitions.

\begin{theorem}[Viscosity Formula]
\label{thm:viscosity}
Fluid with partition lag distribution $\{\taulagij\}$ and coupling strengths $\{g_{ij}\}$ has viscosity
\begin{equation}
\mu = \sum_{i,j} \taulagij g_{ij}
\end{equation}
\end{theorem}

\begin{proof}
This is special case of Theorem \ref{thm:transport} for shear flow. Viscosity measures momentum diffusion perpendicular to flow direction. Momentum transfer requires particle transitions between layers with different velocities. Transition rate $\Gamma_{ij} = g_{ij}/\taulagij$ determines momentum flux. Resistance to flow is $\mu \propto \sum \taulagij g_{ij}$.
\end{proof}

\subsection{Electrical Resistivity Equation}

Current flow from electron partition transitions.

\begin{theorem}[Resistivity Formula]
\label{thm:resistivity}
Conductor with electron density $n$, scattering partition lags $\{\tau_{s,ij}\}$, and coupling $\{g_{ij}\}$ has resistivity
\begin{equation}
\rho = \frac{1}{ne^2}\sum_{i,j} \tau_{s,ij} g_{ij}
\end{equation}
\end{theorem}

\begin{proof}
Resistivity measures voltage per current density: $\rho = E/J$. Current density is $J = nev_d$ where $v_d$ is drift velocity. Electric field accelerates electrons: $eE = mv_d/\tau$ where $\tau$ is scattering time. This gives $\rho = m/(ne^2\tau)$. For partition-based scattering, $\tau^{-1} = \sum_{ij} g_{ij}/\tau_{s,ij}$, yielding stated formula.
\end{proof}

\subsection{Maxwell Thermodynamic Relations}

Cross-derivatives ensure consistency.

\begin{theorem}[Maxwell Relation Constraints]
\label{thm:maxwell_relations}
Thermodynamic potentials satisfy
\begin{align}
\left(\frac{\partial T}{\partial V}\right)_{S,N} &= -\left(\frac{\partial P}{\partial S}\right)_{V,N} \\
\left(\frac{\partial T}{\partial P}\right)_{S,N} &= \left(\frac{\partial V}{\partial S}\right)_{P,N} \\
\left(\frac{\partial S}{\partial V}\right)_{T,N} &= \left(\frac{\partial P}{\partial T}\right)_{V,N}
\end{align}
\end{theorem}

\begin{proof}
Thermodynamic potentials are exact differentials. For internal energy $dU = TdS - PdV + \mu dN$, mixed partial derivatives commute: $\partial^2 U/(\partial S \partial V) = \partial^2 U/(\partial V \partial S)$. This gives first relation. Other relations follow from Helmholtz free energy $F = U - TS$, enthalpy $H = U + PV$, and Gibbs free energy $G = U - TS + PV$.
\end{proof}

These eleven equations—thermodynamic state, transport coefficients, S-entropy boundedness, metabolic GPS, network topology, Poincaré recurrence, protein folding, membrane flux, viscosity, resistivity, Maxwell relations—uniquely determine cellular state when coupled with twelve measurement modalities.
