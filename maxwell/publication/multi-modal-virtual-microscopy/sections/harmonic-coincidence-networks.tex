\section{Harmonic Coincidence Networks}

\subsection{Vibrational Frequency Relationships}

Molecular vibrational modes form networks through harmonic relationships. A molecule with $N$ atoms has $3N-6$ vibrational normal modes (or $3N-5$ for linear molecules), each with characteristic frequency $\omega_j$.

\begin{definition}[Harmonic Coincidence]
Two frequencies $\omega_1$ and $\omega_2$ exhibit harmonic coincidence at harmonic numbers $(n_1, n_2)$ when
\begin{equation}
|n_1\omega_1 - n_2\omega_2| < \Delta\omega_{\text{threshold}}
\end{equation}
where $\Delta\omega_{\text{threshold}}$ is coincidence detection bandwidth.
\end{definition}

For molecular vibrations with $\omega \sim 10^{13}$--$10^{14}$ rad/s, appropriate threshold is $\Delta\omega_{\text{threshold}} = 10^{11}$ Hz ($\approx 3$ cm$^{-1}$), below spectroscopic resolution but above thermal broadening.

\begin{definition}[Harmonic Network]
Harmonic network $\mathcal{H} = (V, E)$ comprises vertices $V$ representing vibrational modes $\{\omega_j\}$ and edges $E$ connecting modes with harmonic coincidences. Edge weights quantify coincidence strength: $w_{ij} = |n_i\omega_i - n_j\omega_j|^{-1}$.
\end{definition}

\begin{figure*}[htbp]
    \centering
    \includegraphics[width=\textwidth]{figures/co2_molecular_demon_lattice.png}
    \caption{\textbf{CO$_2$ Molecular Demon Lattice: 4×4×4 Collective Vibrational States.}
    (A) CO$_2$ molecular demon lattice structure with 64 molecules arranged in 4×4×4 grid showing spatial distribution with color-coded Z-position (0.0--3.0).
    (B) CO$_2$ vibrational modes fundamental frequencies: Mode 1 ($\nu_1$ sym stretch) 40.17 THz, Mode 2 ($\nu_2$ bend) 20.00 THz, Mode 3 ($\nu_2$ bend) 20.00 THz, Mode 4 ($\nu_3$ asym stretch) 70.42 THz.
    (C) Vibrational energy levels quantum state energies: Mode 1 (26.62 zJ), Mode 2 (13.25 zJ), Mode 3 (13.25 zJ), Mode 4 (46.66 zJ).
    (D) Average S-category coordinates collective categorical state showing $s_E = 0.5414$, $s_I = 0.3250$, $s_K = 0.9050$.
    (E) Observation statistics lattice measurement summary: 64 total molecules, 1128 observations, 17.6 obs/molecule, 4 vibrational modes.
    (F) Mode consistency across runs reproducibility check comparing Run 1 (red) vs Run 2 (blue) showing excellent agreement across all four modes.
    (G) Lattice density metrics spatial distribution: 1.0 molecules/site, 17.6 observations/site, 64 total sites.}
    \label{fig:co2_demon_lattice}
\end{figure*}

\subsection{Frequency Space Triangulation}

Harmonic relationships constrain frequency space topology, enabling structure prediction.

\begin{theorem}[Frequency Triangulation]
\label{thm:frequency_triangulation}
Unknown frequency $\omega_*$ connected to at least three known frequencies $\{\omega_1, \omega_2, \omega_3\}$ through harmonic relationships $(n_{*i}, n_{i,*})$ is determined to within coincidence bandwidth through
\begin{equation}
\omega_* = \frac{\sum_{i=1}^{K} w_i \omega_*^{(i)}}{\sum_{i=1}^{K} w_i}
\end{equation}
where $\omega_*^{(i)} = (n_{i,*}/n_{*i})\omega_i$ and $w_i = |n_{*i}\omega_*^{(i)} - n_{i,*}\omega_i|^{-2}$.
\end{theorem}

\begin{proof}
Each harmonic relationship with mode $i$ gives estimate $\omega_*^{(i)} = (n_{i,*}/n_{*i})\omega_i$. With $K \geq 3$ relationships, system is overdetermined. Optimal estimate uses inverse-square weighting to minimize error contribution from weak coincidences. Uncertainty is
\begin{equation}
\sigma_{\omega_*} = \sqrt{\frac{1}{\sum_{i=1}^{K} w_i}} \sim \frac{\Delta\omega_{\text{threshold}}}{\sqrt{K}}
\end{equation}
For $K \geq 3$, prediction accuracy approaches coincidence bandwidth.
\end{proof}

\subsection{Connection to Categorical Framework}

Harmonic networks provide explicit implementation of categorical distance.

\begin{theorem}[Harmonic-Categorical Correspondence]
For vibrational modes with frequencies $\omega_i$ and partition depths $n_i = \lfloor\omega_i/\omega_0\rfloor$, categorical distance approximates harmonic network distance:
\begin{equation}
\dcat(\Sigma_i, \Sigma_j) \approx \min_{(m_i,m_j)} |m_i n_i - m_j n_j|
\end{equation}
where minimum is over harmonic numbers producing coincidence.
\end{theorem}

\begin{proof}
Partition depth $n$ relates to frequency through $E_n = n^2\hbar\omega_0$ (Theorem \ref{thm:frequency_depth}). For mode with frequency $\omega$, depth is $n \sim \sqrt{\omega/\omega_0}$. Categorical distance $\dcat = |n_i - n_j|$ becomes $\dcat \sim |\sqrt{\omega_i} - \sqrt{\omega_j}|/\sqrt{\omega_0}$.

For harmonic coincidence $m_i\omega_i \approx m_j\omega_j$, rearranging gives $\omega_i/\omega_j \approx m_j/m_i$. Substituting into categorical distance:
\begin{equation}
\dcat \sim \frac{1}{\sqrt{\omega_0}}|\sqrt{\omega_i} - \sqrt{\omega_j}| = \frac{\sqrt{\omega_i}}{\sqrt{\omega_0}}\left|1 - \sqrt{\frac{m_i}{m_j}}\right|
\end{equation}

For nearly coincident modes, $m_i/m_j \approx 1$, giving $\dcat \approx |m_i n_i - m_j n_j|/n_i$ up to normalization. Thus harmonic network distance and categorical distance are equivalent measures.
\end{proof}

\subsection{Structure Prediction from Partial Data}

\begin{corollary}[Spectroscopic Completion]
Given $M$ measured vibrational frequencies and their harmonic network, all $3N-6$ modes of an $N$-atom molecule can be predicted when average network connectivity satisfies $\langle k \rangle \geq 3$.
\end{corollary}

\begin{proof}
Each unknown mode requires $K \geq 3$ connections for triangulation (Theorem \ref{thm:frequency_triangulation}). With $M$ known modes and average degree $\langle k \rangle$, total edges are $E = M\langle k \rangle/2$. These edges support prediction of $\sim E/3$ unknown modes. For complete prediction of all $3N-6$ modes from $M$ measurements: $(3N-6-M) \leq M\langle k \rangle/6$, giving $\langle k \rangle \geq 6(3N-6-M)/M$. When $M \gg 1$ and $M \sim (3N-6)/2$, condition reduces to $\langle k \rangle \geq 3$.
\end{proof}

\subsection{Error Sources and Scaling}

Prediction error has two components: triangulation uncertainty and anharmonicity.

\begin{theorem}[Prediction Error Scaling]
Prediction error for unknown mode connected to $K$ known modes among total $M$ measured modes scales as
\begin{equation}
\epsilon(\omega_*) = \sqrt{\frac{\Delta\omega_{\text{threshold}}^2}{K} + \frac{(\chi\langle n \rangle\omega_*)^2}{M}}
\end{equation}
where $\chi \sim 0.01$ is anharmonicity constant and $\langle n \rangle$ is average harmonic number.
\end{theorem}

\begin{proof}
Triangulation uncertainty decreases as $1/\sqrt{K}$ (standard error of mean). Anharmonicity shifts frequencies according to $\omega_{\text{real}} = \omega_0(1 - \chi v)$ where $v$ is vibrational quantum number. For harmonic $n$, effective $v \sim n$, giving error $\chi n \omega$. Averaging over $M$ modes by central limit theorem: $\sigma_{\chi} = \chi\langle n \rangle\omega/\sqrt{M}$. Combining errors in quadrature yields stated formula.
\end{proof}

\begin{figure}[htbp]
    \centering
    \includegraphics[width=\textwidth]{figures/vibration_field_mapper_panel.png}
    \caption{\textbf{Quantum partition structure showing atomic orbitals, vibrational modes, and angular momentum distributions.} 
    \textbf{(A) Negation field map at $Z=1$ (hydrogen).} Electrostatic potential $\phi(\mathbf{r}) = -Z/r = -1/r$ (color map: blue = most negative, red = least negative) creates radial field (white arrows) pointing toward nucleus at origin. Field strength decreases as $1/r^2$, shown by arrow length. At $Z=1$, potential is weakly confining with $\phi \sim -9$ (blue) at nucleus and $\phi \sim 0$ (red) at large radius. This weak confinement produces large atomic radius (Bohr radius $a_0 \sim 0.53$ Å) and low ionization energy ($13.6$ eV). Radial symmetry reflects spherical partition structure of s-orbitals.
    \textbf{(B) Negation field map at $Z=6$ (carbon).} Stronger nuclear charge creates deeper potential well with $\phi \sim -54$ (dark blue) at nucleus. Field arrows are longer near nucleus, indicating stronger confinement. Potential gradient is steeper, producing smaller atomic radius and higher ionization energy ($11.3$ eV for neutral carbon). Concentric circles in potential map show equipotential surfaces. Stronger confinement leads to more tightly bound electrons and higher orbital energies. The $Z=6$ potential supports multiple bound states (1s, 2s, 2p) with distinct radial nodes.
    \textbf{(C) Boundary probability distributions.} Radial probability density $|\psi(r)|^2 r^2$ vs. radius for different quantum states. 1s orbital (blue, $n=1, l=0$) peaks at $r \sim 1$ Bohr radius with no nodes. 2s orbital (green, $n=2, l=0$) has one radial node and peaks at $r \sim 5$ Bohr radii. 2p orbital (orange, $n=2, l=1$) peaks at $r \sim 4$ Bohr radii. 3s orbital (red, $n=3, l=0$) has two nodes and peaks at $r \sim 10$ Bohr radii. Higher $n$ states have larger radial extent and more nodes, reflecting higher energy and more complex partition structure. 
    \textbf{(D) Vibrational modes (harmonic oscillator).} Energy levels $E_\nu = \hbar\omega(\nu + 1/2)$ for quantum harmonic oscillator showing vibrational states $\nu = 0, 1, 2, 3$ (blue, orange, green, red). Ground state $\nu=0$ (blue) has zero-point energy $E_0 = \hbar\omega/2$ and Gaussian wavefunction. Excited states have higher energy and more nodes (oscillations in wavefunction). Classical turning points (where wavefunction amplitude decreases) move outward with increasing $\nu$. Shaded regions show probability density $|\psi_\nu(x)|^2$. Black curve shows classical potential $V(x) = \frac{1}{2}m\omega^2 x^2$. Vibrational partition structure determines molecular IR absorption spectrum.
    \textbf{(E) IR spectrum - partition oscillations.} Infrared transmittance showing vibrational absorption bands. O-H stretch ($\sim 3500$ cm$^{-1}$) appears as sharp dip due to high-frequency O-H bond vibration. C-H stretch ($\sim 3000$ cm$^{-1}$) shows similar feature. C=O stretch ($\sim 1700$ cm$^{-1}$) is strong absorption from carbonyl group. S-H stretch ($\sim 2600$ cm$^{-1}$) appears at intermediate frequency. Each absorption corresponds to transition between vibrational states ($\Delta\nu = 1$), measuring partition oscillation frequency. 
    \textbf{(F) Angular complexity distributions.} Phase space topology showing angular momentum states in complex plane. $l=0$ (s-orbital, blue) is spherically symmetric with no angular nodes. $l=1$ (p-orbital, orange) has one angular node, creating dumbbell shape. $l=2$ (d-orbital, red) has two angular nodes, creating cloverleaf pattern. $l=3$ (f-orbital, purple) has three angular nodes, creating complex rosette pattern. Each lobe represents region of high probability density. Number of lobes increases with angular momentum quantum number $l$. }
    \label{fig:quantum_partition_structure}
    \end{figure}

\subsection{Application to HCNA Modality}

Harmonic Coincidence Network Analysis (Modality 6) uses network topology to extract thermodynamic information.

\begin{theorem}[Network Clustering and Temperature]
Network clustering coefficient $\langle C \rangle$ relates to temperature through
\begin{equation}
\kB T = \frac{\hbar\langle\omega\rangle}{2}\coth^{-1}(\langle C \rangle)
\end{equation}
where $\langle\omega\rangle$ is mean vibrational frequency.
\end{theorem}

\begin{proof}
Clustering coefficient measures probability that neighbors of a node are also neighbors. For thermal system, modes with energies $\Delta E \sim \kB T$ are thermally coupled. Coupling creates network edges when $|\omega_i - \omega_j| < \kB T/\hbar$. Higher temperature increases coupling range, increasing clustering. At temperature $T$, fraction of thermally accessible modes is $f = \exp(-\hbar\omega/\kB T)$. Clustering coefficient approximates $\langle C \rangle \sim f^2$ for random network. Solving: $\kB T \sim \hbar\omega/\ln(1/\sqrt{\langle C \rangle})$. Exact relation includes hyperbolic cotangent from Bose-Einstein statistics.
\end{proof}

This provides independent temperature measurement from network structure alone, contributing exclusion factor $\epsilon_{\text{HCNA}} \sim 10^{-3}$ to multi-modal framework.
