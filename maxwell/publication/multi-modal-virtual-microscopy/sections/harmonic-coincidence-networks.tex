\section{Harmonic Coincidence Networks}

\subsection{Vibrational Frequency Relationships}

Molecular vibrational modes form networks through harmonic relationships. A molecule with $N$ atoms has $3N-6$ vibrational normal modes (or $3N-5$ for linear molecules), each with characteristic frequency $\omega_j$.

\begin{definition}[Harmonic Coincidence]
Two frequencies $\omega_1$ and $\omega_2$ exhibit harmonic coincidence at harmonic numbers $(n_1, n_2)$ when
\begin{equation}
|n_1\omega_1 - n_2\omega_2| < \Delta\omega_{\text{threshold}}
\end{equation}
where $\Delta\omega_{\text{threshold}}$ is coincidence detection bandwidth.
\end{definition}

For molecular vibrations with $\omega \sim 10^{13}$--$10^{14}$ rad/s, appropriate threshold is $\Delta\omega_{\text{threshold}} = 10^{11}$ Hz ($\approx 3$ cm$^{-1}$), below spectroscopic resolution but above thermal broadening.

\begin{definition}[Harmonic Network]
Harmonic network $\mathcal{H} = (V, E)$ comprises vertices $V$ representing vibrational modes $\{\omega_j\}$ and edges $E$ connecting modes with harmonic coincidences. Edge weights quantify coincidence strength: $w_{ij} = |n_i\omega_i - n_j\omega_j|^{-1}$.
\end{definition}

\subsection{Frequency Space Triangulation}

Harmonic relationships constrain frequency space topology, enabling structure prediction.

\begin{theorem}[Frequency Triangulation]
\label{thm:frequency_triangulation}
Unknown frequency $\omega_*$ connected to at least three known frequencies $\{\omega_1, \omega_2, \omega_3\}$ through harmonic relationships $(n_{*i}, n_{i,*})$ is determined to within coincidence bandwidth through
\begin{equation}
\omega_* = \frac{\sum_{i=1}^{K} w_i \omega_*^{(i)}}{\sum_{i=1}^{K} w_i}
\end{equation}
where $\omega_*^{(i)} = (n_{i,*}/n_{*i})\omega_i$ and $w_i = |n_{*i}\omega_*^{(i)} - n_{i,*}\omega_i|^{-2}$.
\end{theorem}

\begin{proof}
Each harmonic relationship with mode $i$ gives estimate $\omega_*^{(i)} = (n_{i,*}/n_{*i})\omega_i$. With $K \geq 3$ relationships, system is overdetermined. Optimal estimate uses inverse-square weighting to minimize error contribution from weak coincidences. Uncertainty is
\begin{equation}
\sigma_{\omega_*} = \sqrt{\frac{1}{\sum_{i=1}^{K} w_i}} \sim \frac{\Delta\omega_{\text{threshold}}}{\sqrt{K}}
\end{equation}
For $K \geq 3$, prediction accuracy approaches coincidence bandwidth.
\end{proof}

\subsection{Connection to Categorical Framework}

Harmonic networks provide explicit implementation of categorical distance.

\begin{theorem}[Harmonic-Categorical Correspondence]
For vibrational modes with frequencies $\omega_i$ and partition depths $n_i = \lfloor\omega_i/\omega_0\rfloor$, categorical distance approximates harmonic network distance:
\begin{equation}
\dcat(\Sigma_i, \Sigma_j) \approx \min_{(m_i,m_j)} |m_i n_i - m_j n_j|
\end{equation}
where minimum is over harmonic numbers producing coincidence.
\end{theorem}

\begin{proof}
Partition depth $n$ relates to frequency through $E_n = n^2\hbar\omega_0$ (Theorem \ref{thm:frequency_depth}). For mode with frequency $\omega$, depth is $n \sim \sqrt{\omega/\omega_0}$. Categorical distance $\dcat = |n_i - n_j|$ becomes $\dcat \sim |\sqrt{\omega_i} - \sqrt{\omega_j}|/\sqrt{\omega_0}$.

For harmonic coincidence $m_i\omega_i \approx m_j\omega_j$, rearranging gives $\omega_i/\omega_j \approx m_j/m_i$. Substituting into categorical distance:
\begin{equation}
\dcat \sim \frac{1}{\sqrt{\omega_0}}|\sqrt{\omega_i} - \sqrt{\omega_j}| = \frac{\sqrt{\omega_i}}{\sqrt{\omega_0}}\left|1 - \sqrt{\frac{m_i}{m_j}}\right|
\end{equation}

For nearly coincident modes, $m_i/m_j \approx 1$, giving $\dcat \approx |m_i n_i - m_j n_j|/n_i$ up to normalization. Thus harmonic network distance and categorical distance are equivalent measures.
\end{proof}

\subsection{Structure Prediction from Partial Data}

\begin{corollary}[Spectroscopic Completion]
Given $M$ measured vibrational frequencies and their harmonic network, all $3N-6$ modes of an $N$-atom molecule can be predicted when average network connectivity satisfies $\langle k \rangle \geq 3$.
\end{corollary}

\begin{proof}
Each unknown mode requires $K \geq 3$ connections for triangulation (Theorem \ref{thm:frequency_triangulation}). With $M$ known modes and average degree $\langle k \rangle$, total edges are $E = M\langle k \rangle/2$. These edges support prediction of $\sim E/3$ unknown modes. For complete prediction of all $3N-6$ modes from $M$ measurements: $(3N-6-M) \leq M\langle k \rangle/6$, giving $\langle k \rangle \geq 6(3N-6-M)/M$. When $M \gg 1$ and $M \sim (3N-6)/2$, condition reduces to $\langle k \rangle \geq 3$.
\end{proof}

\subsection{Error Sources and Scaling}

Prediction error has two components: triangulation uncertainty and anharmonicity.

\begin{theorem}[Prediction Error Scaling]
Prediction error for unknown mode connected to $K$ known modes among total $M$ measured modes scales as
\begin{equation}
\epsilon(\omega_*) = \sqrt{\frac{\Delta\omega_{\text{threshold}}^2}{K} + \frac{(\chi\langle n \rangle\omega_*)^2}{M}}
\end{equation}
where $\chi \sim 0.01$ is anharmonicity constant and $\langle n \rangle$ is average harmonic number.
\end{theorem}

\begin{proof}
Triangulation uncertainty decreases as $1/\sqrt{K}$ (standard error of mean). Anharmonicity shifts frequencies according to $\omega_{\text{real}} = \omega_0(1 - \chi v)$ where $v$ is vibrational quantum number. For harmonic $n$, effective $v \sim n$, giving error $\chi n \omega$. Averaging over $M$ modes by central limit theorem: $\sigma_{\chi} = \chi\langle n \rangle\omega/\sqrt{M}$. Combining errors in quadrature yields stated formula.
\end{proof}

\subsection{Application to HCNA Modality}

Harmonic Coincidence Network Analysis (Modality 6) uses network topology to extract thermodynamic information.

\begin{theorem}[Network Clustering and Temperature]
Network clustering coefficient $\langle C \rangle$ relates to temperature through
\begin{equation}
\kB T = \frac{\hbar\langle\omega\rangle}{2}\coth^{-1}(\langle C \rangle)
\end{equation}
where $\langle\omega\rangle$ is mean vibrational frequency.
\end{theorem}

\begin{proof}
Clustering coefficient measures probability that neighbors of a node are also neighbors. For thermal system, modes with energies $\Delta E \sim \kB T$ are thermally coupled. Coupling creates network edges when $|\omega_i - \omega_j| < \kB T/\hbar$. Higher temperature increases coupling range, increasing clustering. At temperature $T$, fraction of thermally accessible modes is $f = \exp(-\hbar\omega/\kB T)$. Clustering coefficient approximates $\langle C \rangle \sim f^2$ for random network. Solving: $\kB T \sim \hbar\omega/\ln(1/\sqrt{\langle C \rangle})$. Exact relation includes hyperbolic cotangent from Bose-Einstein statistics.
\end{proof}

This provides independent temperature measurement from network structure alone, contributing exclusion factor $\epsilon_{\text{HCNA}} \sim 10^{-3}$ to multi-modal framework.
