\section{Current Flow Constraints}

\subsection{Electrical Conduction from Electron Transitions}

Cellular structures conduct ions and electrons through membranes and cytoplasm. This section derives current flow from partition-based electron state transformations.

\subsection{Ohm's Law from Partition Scattering}

\begin{theorem}[Ohm's Law Derivation]
\label{thm:ohm_law}
Conductor with resistivity $\rho = \sum_{ij} \tau_{s,ij} g_{ij}/(ne^2)$ satisfies Ohm's law
\begin{equation}
\mathbf{J} = \sigma \mathbf{E} = \frac{1}{\rho}\mathbf{E}
\end{equation}
where $\mathbf{J}$ is current density and $\mathbf{E}$ is electric field.
\end{theorem}

\begin{proof}
Electric field $\mathbf{E}$ accelerates electrons with charge $-e$ and mass $m$: $\mathbf{F} = -e\mathbf{E} = m\mathbf{a}$. In steady state, acceleration balances scattering: $v_d = (eE/m)\tau_s$ where $\tau_s$ is scattering time and $v_d$ is drift velocity.

Current density is $\mathbf{J} = -ne\mathbf{v}_d$ where $n$ is electron density. Substituting drift velocity:
\begin{equation}
\mathbf{J} = -ne \times \left(-\frac{e\mathbf{E}}{m}\tau_s\right) = \frac{ne^2\tau_s}{m}\mathbf{E}
\end{equation}

For partition-based scattering, scattering rate is $\tau_s^{-1} = \sum_{ij} g_{ij}/\tau_{s,ij}$. Inverting:
\begin{equation}
\tau_s = \left(\sum_{ij} \frac{g_{ij}}{\tau_{s,ij}}\right)^{-1} \approx \frac{\sum_{ij} \tau_{s,ij} g_{ij}}{\sum_{ij} g_{ij}}
\end{equation}

For $\sum_{ij} g_{ij} \sim ne^2/m$, this gives $\rho = m\tau_s/(ne^2) = \sum_{ij} \tau_{s,ij} g_{ij}/(ne^2)$, yielding Ohm's law $\mathbf{J} = \mathbf{E}/\rho$.
\end{proof}

\subsection{Conductivity from Network Topology}

\begin{theorem}[Conductivity Formula]
Electrical conductivity relates to phase-lock network through
\begin{equation}
\sigma = \frac{ne^2}{m} \sum_{\text{paths}} \frac{1}{\sum_{(ij) \in \text{path}} \tau_{s,ij}/g_{ij}}
\end{equation}
where sum is over all conducting paths through network.
\end{theorem}

\begin{proof}
Current flows through parallel paths in phase-lock network. Each path $\gamma$ has resistance $R_{\gamma} = \sum_{(ij) \in \gamma} \tau_{s,ij}/g_{ij}$. Parallel resistances combine as $1/R_{\text{total}} = \sum_{\gamma} 1/R_{\gamma}$. Total conductance relates to conductivity through geometry: $\sigma = (L/A) \times (1/R_{\text{total}})$ where $L$ is length and $A$ is cross-sectional area.
\end{proof}

\subsection{Maxwell's Equations from S-Entropy Transformations}

Electromagnetic fields emerge from S-entropy coordinate gradients.

\begin{theorem}[Electric Field from S-Entropy]
Electric field $\mathbf{E}$ relates to S-entropy gradient through
\begin{equation}
\mathbf{E} = -\lambda_{\text{EM}} \nabla \Sk
\end{equation}
where $\lambda_{\text{EM}} = \hbar\omega_0/e$ is electromagnetic coupling constant.
\end{theorem}

\begin{proof}
Electric potential difference drives charge redistribution, changing knowledge entropy $\Sk$. Potential is $\phi = \lambda_{\text{EM}} \Sk$. Electric field is $\mathbf{E} = -\nabla\phi = -\lambda_{\text{EM}} \nabla\Sk$.
\end{proof}

\begin{theorem}[Magnetic Field from S-Entropy]
Magnetic field $\mathbf{B}$ relates to S-entropy circulation:
\begin{equation}
\mathbf{B} = \lambda_{\text{EM}} \nabla \times (\St\hat{\boldsymbol{\theta}})
\end{equation}
where $\hat{\boldsymbol{\theta}}$ is angular direction associated with temporal entropy $\St$.
\end{theorem}

\begin{proof}
Magnetic field arises from moving charges (currents). Current density $\mathbf{J} = \sigma\mathbf{E}$ creates circulation of temporal entropy $\St$. Ampère's law $\nabla \times \mathbf{B} = \mu_0 \mathbf{J}$ combined with $\mathbf{J} \propto \nabla\Sk$ gives $\mathbf{B} \propto \nabla \times \mathbf{A}$ where vector potential $\mathbf{A}$ relates to $\St$.
\end{proof}

\begin{figure*}[htbp]
    \centering
    \includegraphics[width=\textwidth]{figures/panel_electric_field_mechanics.png}
    \caption{\textbf{Electromagnetic Field Mechanics and Electron Trajectories.} 
    \textit{Top row:} (\textbf{Left}) Electric field configuration around two opposite charges ($+$ and $-$). Field lines (blue arrows) emanate from positive charges and terminate on negative charges. The cyan circle shows an equipotential surface. (\textbf{Center}) Magnetic field (wire cross-section) around a current-carrying wire. The magnetic field forms concentric circles (blue arrows) around the wire (yellow circle at center). (\textbf{Right}) Electron trajectories in three-dimensional space under combined electric and magnetic fields. Colored trajectories show helical motion characteristic of charged particles in crossed fields. 
    \textit{Middle row:} (\textbf{Left}) Newton's cradle model showing resistance as damping. Three curves show wave propagation in superconductor ($R = 0$, green, no damping), medium-$R$ material (yellow, moderate damping), and high-$R$ material (red, strong damping). Resistance damps the wave amplitude exponentially. (\textbf{Center}) Potential landscape showing the S-coordinate potential $\Phi(x, y)$ as a three-dimensional surface. Peaks represent high-potential regions; valleys represent low-potential regions. Current flows from high to low potential. (\textbf{Right}) Material resistance comparison on a logarithmic scale. Superconductors (germanium) have resistivity $\rho < 10^{-100}~\Omega\cdot$m below $T_c$. Metals (copper, aluminum, tungsten) have $\rho \sim 10^{-8}$ to $10^{-5}~\Omega\cdot$m. Nichrome (resistor alloy) has $\rho \sim 10^{-6}~\Omega\cdot$m. The resistivity spans over 100 orders of magnitude.}
    \label{fig:em_field_mechanics}
    \end{figure*}
\subsection{Membrane Potential from Ion Channels}

Cell membranes maintain potential difference through selective ion transport.

\begin{theorem}[Nernst Potential]
Membrane potential for ion species $i$ with concentrations $c_{\text{in}}$ and $c_{\text{out}}$ is
\begin{equation}
V_i = \frac{\kB T}{z_i e}\ln\left(\frac{c_{\text{out}}}{c_{\text{in}}}\right)
\end{equation}
where $z_i$ is valence.
\end{theorem}

\begin{proof}
At equilibrium, electrochemical potential is uniform: $\mu_{\text{in}} + z_i e V_{\text{in}} = \mu_{\text{out}} + z_i e V_{\text{out}}$. Chemical potential for ideal solution: $\mu = \mu_0 + \kB T \ln c$. Substituting and solving for potential difference $V = V_{\text{in}} - V_{\text{out}}$ yields Nernst equation.
\end{proof}

\begin{corollary}
For potassium with $c_{\text{in}} = 140$ mM, $c_{\text{out}} = 5$ mM at $T = 310$ K:
\begin{equation}
V_{K} = \frac{26.7\text{ mV}}{1}\ln\left(\frac{5}{140}\right) = -88\text{ mV}
\end{equation}
\end{corollary}

\subsection{Goldman-Hodgkin-Katz Equation}

Multiple ion species contribute to membrane potential.

\begin{theorem}[GHK Equation]
Membrane potential with multiple ion species is
\begin{equation}
V_m = \frac{\kB T}{e}\ln\left(\frac{\sum_i P_i c_i^{\text{out}}}{\sum_i P_i c_i^{\text{in}}}\right)
\end{equation}
where $P_i$ is permeability for ion $i$.
\end{theorem}

\begin{proof}
Total current is sum of individual ion currents: $I = \sum_i I_i$. At steady state, $I = 0$. Each ion current is $I_i = P_i (c_i^{\text{in}} - c_i^{\text{out}} e^{eV/\kB T})$. Setting $\sum_i I_i = 0$ and solving for $V$ yields GHK equation.
\end{proof}

\subsection{Current-Voltage Relationship}

\begin{theorem}[I-V Characteristic]
Ion channel with open probability $\alpha$ and single-channel conductance $\gamma$ has current
\begin{equation}
I = N_T \alpha \gamma (V - V_{\text{rev}})
\end{equation}
where $N_T$ is total channel number and $V_{\text{rev}}$ is reversal potential.
\end{theorem}

\begin{proof}
Single open channel passes current $i = \gamma(V - V_{\text{rev}})$ where $V_{\text{rev}}$ is Nernst potential for that ion. Number of open channels is $N_{\text{open}} = N_T \alpha$. Total current is $I = N_{\text{open}} \times i$.
\end{proof}

\subsection{Dimensional Reduction for Current Flow}

\begin{theorem}[Zero-Dimensional Cross-Section]
Current flow through conductor reduces to zero-dimensional cross-section (number of parallel paths) plus one-dimensional S-coordinate along conductor.
\end{theorem}

\begin{proof}
Phase-lock coupling enforces categorical coherence across conductor cross-section. All electrons in cross-section occupy same partition state. Current flow is one-dimensional along conductor length, parameterized by S-coordinate transformation. Number of conducting channels (parallel paths) is discrete: $N_{\text{channels}}$. Total conductance is $G = N_{\text{channels}} \times g_{\text{single}}$.

This reduces $\sim 10^{23}$ electron degrees of freedom to one collective state plus channel count.
\end{proof}

\subsection{Resistivity Temperature Dependence}

\begin{theorem}[Metallic Resistivity]
For metallic conductors, resistivity increases with temperature:
\begin{equation}
\rho(T) = \rho_0[1 + \alpha_T(T - T_0)]
\end{equation}
where $\alpha_T$ is temperature coefficient.
\end{theorem}

\begin{proof}
From $\rho = \sum_{ij} \tau_{s,ij} g_{ij}/(ne^2)$, partition scattering time $\tau_{s,ij}$ decreases with temperature as phonon scattering increases: $\tau_s^{-1} \propto T$ for $T > \Theta_D$ (Debye temperature). This gives linear resistivity increase.
\end{proof}

\subsection{Application to Cellular Conduction}

Cellular electrical properties constrain structure.

\begin{example}[Membrane Capacitance]
Cell membrane with thickness $d \sim 5$ nm, area $A \sim 10^{-9}$ m$^2$, and dielectric constant $\epsilon_r \sim 3$ has capacitance
\begin{equation}
C = \frac{\epsilon_0 \epsilon_r A}{d} = \frac{8.85 \times 10^{-12} \times 3 \times 10^{-9}}{5 \times 10^{-9}} \sim 5.3\text{ pF}
\end{equation}
\end{example}

\begin{example}[Membrane Resistance]
Ion channel with conductance $\gamma \sim 10$ pS and density $\rho_{\text{channel}} \sim 1$ $\mu$m$^{-2}$ gives membrane resistance
\begin{equation}
R = \frac{1}{A \rho_{\text{channel}} \gamma} = \frac{1}{10^{-9} \times 10^{12} \times 10^{-11}} \sim 10^8\text{ }\Omega
\end{equation}
\end{example}

\subsection{Constraint Application}

Current flow constraints eliminate structures with inconsistent electrical properties.

\begin{theorem}[Electrical Exclusion]
Structure $S$ must satisfy:
\begin{enumerate}
\item Predicted membrane potential $V_{\text{pred}}(S)$ matches measured $V_{\text{meas}}$ within $\pm 5$ mV
\item Predicted capacitance $C_{\text{pred}}(S)$ matches measured $C_{\text{meas}}$ within $\pm 10\%$
\item Predicted resistance $R_{\text{pred}}(S)$ matches measured $R_{\text{meas}}$ within $\pm 20\%$
\end{enumerate}
Failure on any criterion excludes structure.
\end{theorem}

With three independent electrical measurements, exclusion factor is $\epsilon_{\text{electrical}} \sim 10^{-3}$. This provides independent validation of membrane structure, ion channel density, and cytoplasmic composition.
