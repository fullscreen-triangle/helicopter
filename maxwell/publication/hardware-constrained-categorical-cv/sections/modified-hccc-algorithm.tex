\section{Modified HCCC Algorithm with Dual-Membrane Integration}

\subsection{Algorithm Overview}

The modified hardware-constrained categorical completion algorithm integrates pixel Maxwell demons, dual-membrane structure, hierarchical network BMDs, and hardware stream coherence into a unified image understanding framework.

\begin{algorithm}[H]
\caption{Dual-Membrane Hardware-Constrained Categorical Completion}
\label{alg:dual_membrane_hccc}
\begin{algorithmic}[1]
\STATE \textbf{Input:} Image $I$, Transform type $T$
\STATE \textbf{Output:} Dual network BMD $\beta^{(\text{network})}_{\text{dual,final}}$, Sequence $\sigma$, Depth map $D$
\STATE
\STATE // \textit{Initialize pixel demon grid from image}
\STATE $G_{\text{pixel}} \leftarrow \text{DualMembraneGrid.from\_image}(I, T)$
\STATE $G_{\text{pixel}}.\text{initialize\_atmospheric\_lattice}()$
\STATE
\STATE // \textit{Measure hardware BMD stream (zero backaction)}
\STATE $\beta^{(\text{stream})}_{\text{hardware}} \leftarrow \text{MeasureHardwareStream}()$
\STATE $\beta^{(\text{stream})}_{\text{pixel}} \leftarrow G_{\text{pixel}}.\text{measure\_grid\_stream}()$
\STATE $\beta^{(\text{stream})}_{\text{complete}} \leftarrow \beta^{(\text{stream})}_{\text{hardware}} \circledast \beta^{(\text{stream})}_{\text{pixel}}$
\STATE
\STATE // \textit{Initialize dual network BMD}
\STATE $\beta^{(\text{network})}_{\text{dual},0} \leftarrow \text{DualNetworkBMD}(\beta^{(\text{stream})}_{\text{complete}}, T)$
\STATE
\STATE // \textit{Segment image into dual-membrane regions}
\STATE $\mathcal{R}_{\text{available}} \leftarrow \text{SegmentIntoDualRegions}(I, G_{\text{pixel}})$
\STATE $\mathcal{R}_{\text{processed}} \leftarrow \emptyset$
\STATE $\sigma \leftarrow ()$
\STATE $i \leftarrow 0$
\STATE
\WHILE{$\mathcal{R}_{\text{available}} \neq \emptyset$}
    \STATE
    \STATE // \textit{Update hardware stream (continuous measurement)}
    \STATE $\beta^{(\text{stream})}_{\text{complete}} \leftarrow \text{UpdateHardwareStream}()$
    \STATE
    \STATE // \textit{Select region by stream-coherent ambiguity}
    \STATE $R_{\text{next}} \leftarrow \arg\max_{R \in \mathcal{R}_{\text{available}}} \left[A(\beta^{(\text{network})}_{\text{dual},i}, R) - \lambda \cdot D_{\text{stream}}(\beta^{(\text{network})}_{\text{dual},i} \circledast R, \beta^{(\text{stream})}_{\text{complete}})\right]$
    \STATE
    \STATE // \textit{Compute observable face ambiguity}
    \STATE $A_{i+1} \leftarrow A(\beta^{(\text{network})}_{\text{dual},i,\text{obs}}, R_{\text{next,obs}})$
    \STATE
    \STATE // \textit{Check termination: coherence achieved}
    \IF{$A_{i+1} < A_{\text{coherence}}$}
        \STATE \textbf{break}
    \ENDIF
    \STATE
    \STATE // \textit{Generate dual BMD through categorical completion}
    \STATE $\beta_{\text{dual},i+1} \leftarrow \text{GenerateDualBMD}(\beta^{(\text{network})}_{\text{dual},i,\text{obs}}, R_{\text{next}}, T)$
    \STATE
    \STATE // \textit{Integrate hierarchically into dual network BMD}
    \STATE $\beta^{(\text{network})}_{\text{dual},i+1} \leftarrow \text{IntegrateHierarchicalDual}(\beta^{(\text{network})}_{\text{dual},i}, \beta_{\text{dual},i+1}, \sigma \cup \{R_{\text{next}}\})$
    \STATE
    \STATE // \textit{Update sequence and region tracking}
    \STATE $\sigma \leftarrow \sigma \cup \{R_{\text{next}}\}$
    \STATE $\mathcal{R}_{\text{processed}} \leftarrow \mathcal{R}_{\text{processed}} \cup \{R_{\text{next}}\}$
    \STATE $\mathcal{R}_{\text{available}} \leftarrow \mathcal{R}_{\text{available}} \setminus \{R_{\text{next}}\}$
    \STATE
    \STATE // \textit{Check revisitation via network-induced ambiguity increase}
    \FOR{$R' \in \mathcal{R}_{\text{processed}}$}
        \STATE $A_{\text{prev}} \leftarrow \text{GetProcessingAmbiguity}(R', \sigma)$
        \STATE $A_{\text{current}} \leftarrow A(\beta^{(\text{network})}_{\text{dual},i+1}, R')$
        \IF{$A_{\text{current}} > A_{\text{prev}} + \Delta A_{\text{revisit}}$}
            \STATE $\mathcal{R}_{\text{available}} \leftarrow \mathcal{R}_{\text{available}} \cup \{R'\}$
            \STATE $\mathcal{R}_{\text{processed}} \leftarrow \mathcal{R}_{\text{processed}} \setminus \{R'\}$
        \ENDIF
    \ENDFOR
    \STATE
    \STATE $i \leftarrow i + 1$
\ENDWHILE
\STATE
\STATE // \textit{Extract depth from membrane thickness}
\STATE $D \leftarrow \text{ExtractDepthMap}(\beta^{(\text{network})}_{\text{dual},i})$
\STATE
\STATE \RETURN $\beta^{(\text{network})}_{\text{dual},i}$, $\sigma$, $D$
\end{algorithmic}
\end{algorithm}

\subsection{Key Algorithm Operations}

\subsubsection{Dual-Membrane Grid Initialization}

The pixel demon grid is created from the image with dual-membrane structure:
\begin{equation}
G_{\text{pixel}} = \{\text{DMPMD}_{ij} : i \in [0, N_x-1], j \in [0, N_y-1]\}
\end{equation}

Each pixel demon is initialized from image intensity $I[i,j]$ through molecular demon lattice creation:
\begin{equation}
S_{k,\text{front}}[i,j] = f(I[i,j]), \quad S_{k,\text{back}}[i,j] = T(S_{k,\text{front}}[i,j])
\end{equation}

\subsubsection{Hardware Stream Measurement}

The complete hardware stream composes external hardware with pixel demon measurements:
\begin{equation}
\beta^{(\text{stream})}_{\text{complete}} = \underbrace{\beta_{\text{display}} \circledast \beta_{\text{network}} \circledast \cdots}_{\text{external hardware}} \circledast \underbrace{\bigoplus_{ij} \beta_{ij}^{\text{pixel}}}_{\text{pixel demons}}
\end{equation}

Zero-backaction pixel demon queries enable continuous stream measurement without system disturbance (Theorem \ref{thm:zero_backaction}).

\subsubsection{Dual Region Segmentation}

Image segmentation creates dual-membrane regions:
\begin{equation}
\mathcal{R} = \{R_1^{\text{dual}}, R_2^{\text{dual}}, \ldots, R_n^{\text{dual}}\}
\end{equation}

Each region $R^{\text{dual}}$ contains:
\begin{itemize}
\item Pixel demon sub-grid $\{PM D_{ij} : (i,j) \in R\}$
\item Observable face indicator $F_R$
\item Front and back categorical states from pixel aggregation
\end{itemize}

\subsubsection{Stream-Coherent Region Selection}

The dual objective balances ambiguity maximization and stream coherence (Definition \ref{def:stream_coherent_ambiguity}):
\begin{equation}
R_{\text{next}} = \arg\max_{R} \left[A(\beta^{(\text{network})}_{\text{dual}}, R) - \lambda D_{\text{stream}}(\beta^{(\text{network})}_{\text{dual}} \circledast R, \beta^{(\text{stream})})\right]
\end{equation}

\textbf{Ambiguity term} $A(\beta^{(\text{network})}_{\text{dual}}, R)$ drives exploration of high categorical richness regions (Theorem \ref{thm:ambiguity_richness}).

\textbf{Stream divergence term} $D_{\text{stream}}$ enforces physical realizability (Theorem \ref{thm:multimodal_coherence}).

The parameter $\lambda$ balances exploration versus constraint satisfaction. Typical values: $\lambda \in [0.3, 0.7]$.

\subsubsection{Dual BMD Generation}

Categorical completion generates dual-membrane BMD states:
\begin{equation}
\beta_{\text{dual},new} = \langle \beta_{\text{front},new}, \beta_{\text{back},new}, F, T \rangle
\end{equation}

where:
\begin{align}
\beta_{\text{front},new} &= \text{Complete}(\beta^{(\text{network})}_{\text{obs}}, R_{\text{obs}}) \\
\beta_{\text{back},new} &= T(\beta_{\text{front},new})
\end{align}

The front face completion selects one weak force configuration from $\mathcal{H}(c_{\text{current}})$ compatible with region constraints. The back face is derived through conjugate transformation, maintaining dual structure.

\subsubsection{Hierarchical Dual Integration}

The new dual BMD integrates into the network BMD hierarchically:
\begin{equation}
\beta^{(\text{network})}_{\text{dual},i+1} = \text{IntegrateHierarchicalDual}(\beta^{(\text{network})}_{\text{dual},i}, \beta_{\text{dual},new}, \sigma)
\end{equation}

This operation (detailed in Section 4.4):
\begin{enumerate}
\item Generates compound BMDs with all previously processed regions
\item Propagates constraints through phase-lock coupling
\item Updates global network BMD through hierarchical composition
\item Maintains conjugate relationship $\beta_{\text{back}} = T(\beta_{\text{front}})$ at all levels
\end{enumerate}

\subsubsection{Network-Induced Revisitation}

As the network evolves, previously processed regions can increase in ambiguity (Definition \ref{def:revisitation_ambiguity}):
\begin{equation}
\text{Revisit } R' \iff A(\beta^{(\text{network})}_i, R') > A(\beta^{(\text{network})}_j, R') + \Delta A_{\text{revisit}}
\end{equation}

where $R'$ was processed at step $j < i$. The threshold $\Delta A_{\text{revisit}}$ prevents oscillation by requiring significant ambiguity increase.

\subsection{Convergence Analysis}

\begin{theorem}[Finite Convergence]
\label{thm:finite_convergence}
Under hardware-constrained categorical completion, the algorithm converges to network coherence in finite iterations:
\begin{equation}
i_{\max} \leq |\mathcal{R}| \cdot \left\lceil \log_2\left(\frac{A_{\text{initial}}}{A_{\text{coherence}}}\right) \right\rceil \cdot (1 + \alpha N_{\text{revisit}})
\end{equation}
where $|\mathcal{R}|$ is region count, $N_{\text{revisit}}$ is expected revisitations per region, and $\alpha$ accounts for path dependence.
\end{theorem}

\begin{proof}
\textbf{Step 1: Per-region ambiguity reduction.} Each comparison with region $R$ generates local BMD completing specific oscillatory holes:
\begin{equation}
A(\beta_R, R) < A(\beta_0, R)
\end{equation}

\textbf{Step 2: Network integration bounds.} Integrating $\beta_R$ into network BMD adds hierarchical structure constrained by hardware noise floor:
\begin{equation}
A(\beta^{(\text{network})}, R) \geq A(\beta_R, R) - k_B T \log(N_{\text{connections}})
\end{equation}

\textbf{Step 3: Connection saturation.} Categorical connections from $R$ saturate at richness limit:
\begin{equation}
N_{\text{connections}} \leq R(R) < \infty
\end{equation}

\textbf{Step 4: Logarithmic revisitation.} Each revisit provides diminishing returns:
\begin{equation}
N_{\text{revisit}}(R) \leq \log_2(R(R)/R(\beta_{\text{final}}^R))
\end{equation}

\textbf{Step 5: Global bound.} With $|\mathcal{R}|$ regions:
\begin{equation}
i_{\max} = |\mathcal{R}| \cdot \lceil \log_2(A_{\text{initial}}/A_{\text{coherence}}) \rceil \cdot (1 + \alpha N_{\text{revisit}})
\end{equation}

All terms finite under hardware precision constraints. $\square$
\end{proof}

For typical parameters ($|\mathcal{R}| = 100$ regions, $A_{\text{initial}}/A_{\text{coherence}} = 10^6$, $N_{\text{revisit}} = 2$):
\begin{equation}
i_{\max} \leq 100 \cdot 20 \cdot 3 = 6000 \text{ iterations}
\end{equation}

In practice, convergence occurs much faster ($i \sim 200-500$) due to exponential ambiguity reduction.

\subsection{Computational Complexity}

\textbf{Per-iteration cost:}
\begin{align}
T_{\text{iteration}} &= \underbrace{\mathcal{O}(|\mathcal{R}| \cdot m)}_{\text{ambiguity calc.}} + \underbrace{\mathcal{O}(m)}_{\text{stream update}} + \underbrace{\mathcal{O}(N_x N_y m)}_{\text{pixel queries}} + \underbrace{\mathcal{O}(2^n_{\text{kept}})}_{\text{network integration}} \\
&\approx \mathcal{O}(|\mathcal{R}| \cdot m + N_x N_y m)
\end{align}

where $m \approx 5$ is molecular species count and $2^n_{\text{kept}} \ll 2^n$ due to compound pruning.

\textbf{Total algorithm cost:}
\begin{equation}
T_{\text{total}} = i_{\max} \cdot T_{\text{iteration}} \approx \mathcal{O}(i_{\max} \cdot |\mathcal{R}| \cdot m + i_{\max} \cdot N_x N_y m)
\end{equation}

For $i_{\max} = 500$, $|\mathcal{R}| = 100$, $N_x = N_y = 512$, $m = 5$:
\begin{equation}
T_{\text{total}} \sim 500 \cdot 100 \cdot 5 + 500 \cdot 512 \cdot 512 \cdot 5 \sim 2.5 \times 10^5 + 6.5 \times 10^8 \sim 6.5 \times 10^8 \text{ operations}
\end{equation}

At 1 GHz (10$^9$ operations/second): $T_{\text{total}} \sim 0.65$ seconds, enabling real-time performance.

The key complexity reduction: $\mathcal{O}(m)$ per pixel query instead of $\mathcal{O}(N_{\text{molecules}} \sim 10^{25})$ through harmonic coincidence networks (Theorem \ref{thm:constant_time_query}).

\subsection{Depth Extraction from Membrane Thickness}

Categorical depth emerges from front-back state separation:
\begin{equation}
D[i,j] = d_S(\mathbf{S}_{\text{front}}[i,j], \mathbf{S}_{\text{back}}[i,j])
\end{equation}

For phase conjugate transformation:
\begin{equation}
D[i,j] = |S_{k,\text{front}}[i,j] - S_{k,\text{back}}[i,j]| = |S_{k,\text{front}}[i,j] - (-S_{k,\text{front}}[i,j])| = 2|S_{k,\text{front}}[i,j]|
\end{equation}

The depth map is extracted directly from network BMD categorical state without geometric reconstruction, stereo correspondence, or depth sensors. High $|S_k|$ indicates thick membrane (high depth); low $|S_k|$ indicates thin membrane (low depth).

\subsection{Face Switching for Complementary Access}

During processing, the algorithm can switch observable faces to access hidden information:
\begin{equation}
\text{If } \text{ShouldSwitch}(\beta^{(\text{network})}_{\text{dual}}, R) \Rightarrow F \leftarrow \bar{F}
\end{equation}

The switching criterion evaluates whether hidden face provides better categorical alignment:
\begin{equation}
\text{ShouldSwitch} \equiv A(\beta^{(\text{network})}_{\text{hidden}}, R) < A(\beta^{(\text{network})}_{\text{obs}}, R) - \epsilon_{\text{switch}}
\end{equation}

Switching accesses complementary categorical projections, analogous to requiring both voltage and current measurements for complete electrical circuit characterization.

\subsection{S-Distance Minimization Implementation}

The algorithm implements S-distance minimization in tri-dimensional S-space $\mathcal{S} = \mathcal{S}_k \times \mathcal{S}_t \times \mathcal{S}_e$ through dual-objective navigation.

\begin{theorem}[HCCC as S-Minimization]
\label{thm:hccc_s_minimization}
The dual-membrane HCCC algorithm implements S-distance minimization dynamics:
\begin{equation}
\frac{d\mathbf{s}}{dt} = -\alpha \nabla_{\mathcal{S}} S(\mathbf{s}, \mathbf{s}^*) - \beta \int_0^t F_{\text{feedback}}(\tau) d\tau + \gamma \xi(t)
\end{equation}
through the correspondence:
\begin{align}
-\alpha \nabla_{\mathcal{S}} S &\leftrightarrow \max A(\beta^{(\text{network})}, R) \quad \text{(exploration in } S_k \text{)} \\
-\beta \int F_{\text{feedback}} &\leftrightarrow -\lambda D_{\text{stream}} \quad \text{(constraint in } S_e \text{)} \\
\gamma \xi(t) &\leftrightarrow \text{revisitation} \quad \text{(stochastic perturbation)}
\end{align}
\end{theorem}

\begin{proof}
The gradient term $-\alpha \nabla_{\mathcal{S}} S$ drives toward lower S-distance. In the knowledge dimension $S_k$, this paradoxically means initially \emph{increasing} categorical richness (ambiguity) to explore solution manifolds. The algorithm implements this through $\max A(\beta^{(\text{network})}, R)$: selecting high-ambiguity regions explores high-$S_k$ space, discovering categorical structure.

The feedback term $-\beta \int F_{\text{feedback}}$ provides environmental coupling constraining exploration to physically realizable states. The hardware stream coherence $-\lambda D_{\text{stream}}$ implements this by penalizing interpretations violating multi-modal hardware measurements, enforcing low $S_e$ (thermodynamic accessibility).

The stochastic term $\gamma \xi$ enables escaping local minima. Revisitation provides controlled perturbation: reconsidering processed regions when network evolution increases ambiguity implements exploration of alternative categorical pathways.

The three terms balance $S_k$ exploration, $S_e$ constraint satisfaction, and $S_t$ temporal evolution, precisely matching S-minimization dynamics. $\square$
\end{proof}

\subsection{Local Termination vs Perpetual Evolution}

The algorithm achieves local termination for specific images while maintaining perpetual network evolution.

\textbf{Local termination} occurs when:
\begin{equation}
A(\beta^{(\text{network})}, R) < A_{\text{coherence}} \quad \forall R \in \mathcal{R}_{\text{available}}
\end{equation}

This does not eliminate ambiguity---it reduces ambiguity below hardware noise floor where further disambiguation is physically impossible.

\textbf{Perpetual evolution} continues: processing the next image starts from $\beta^{(\text{network})}_{\text{current}}$ rather than resetting:
\begin{equation}
\beta^{(\text{network})}_{\text{next image}} = \beta^{(\text{network})}_{\text{current image final}} \circledast \beta_{\text{new hardware stream}}
\end{equation}

Compound BMDs persist across images, enabling cross-image categorical transfer. The network BMD grows perpetually, accumulating processing history indefinitely while individual image sessions achieve local convergence.

\subsection{Energy Dissipation}

Each categorical completion dissipates energy according to Landauer's principle:
\begin{equation}
E_{\text{fill}} \geq k_B T \log N(c)
\end{equation}

where $N(c)$ is the number of weak force configurations at oscillatory hole $c$. For oxygen-based categorical completion with $N \sim 10^6$ configurations:
\begin{equation}
E_{\text{per completion}} \sim k_B T \log(10^6) \sim 10^{-20} \text{ J at } T = 310 \text{ K}
\end{equation}

For $n = 100$ regions:
\begin{equation}
E_{\text{total}} \sim 100 \times 10^{-20} \text{ J} = 10^{-18} \text{ J}
\end{equation}

This matches biological vision energy budgets from oxygen consumption in visual cortex, validating the thermodynamic consistency of the framework.

