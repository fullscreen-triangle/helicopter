\section{Degenerate Matter Equation of State}
\label{sec:degenerate}

We derive the equation of state for degenerate fermionic matter from partition exclusion principles. At low temperatures or high densities, thermal energy $\kB T$ becomes small compared to the Fermi energy $E_F$, and quantum exclusion dominates the partition structure.

\subsection{Partition Exclusion and the Pauli Principle}

For fermions (electrons, neutrons, protons), the Pauli exclusion principle restricts partition occupancy: each partition state $(n,\ell,m,s)$ can be occupied by at most one particle. This is a categorical constraint on partition structure.

At temperature $T = 0$, particles fill partition states from lowest energy upward until all $N$ particles are accommodated. The highest occupied energy defines the Fermi energy $E_F$.

\subsection{Fermi Energy from Partition Capacity}

From Theorem~\ref{thm:capacity}, the number of states with partition depth $\leq n$ is:
\begin{equation}
\mathcal{N}(n) = \sum_{n'=1}^{n} C(n') = \sum_{n'=1}^{n} 2(n')^2 = \frac{2n(n+1)(2n+1)}{6} \approx \frac{2n^3}{3}
\end{equation}

For $N$ particles in volume $V$, the Fermi partition depth $n_F$ satisfies:
\begin{equation}
\mathcal{N}(n_F) = N
\end{equation}

Therefore:
\begin{equation}
n_F = \left(\frac{3N}{2}\right)^{1/3}
\end{equation}

The Fermi energy is the energy of state $n_F$. For a particle in a cubic box with side length $L = V^{1/3}$:
\begin{equation}
E_F = \frac{\hbar^2\pi^2 n_F^2}{2mL^2} = \frac{\hbar^2\pi^2}{2m}\left(\frac{3N}{2V}\right)^{2/3}
\end{equation}

Defining the Fermi wavevector $k_F = (3\pi^2 n)^{1/3}$ where $n = N/V$ is the number density:
\begin{equation}
E_F = \frac{\hbar^2 k_F^2}{2m}
\end{equation}

\subsection{Non-Relativistic Degenerate Electron Gas}

For a non-relativistic electron gas at $T = 0$, the pressure is computed from the energy density. The total energy is:
\begin{equation}
U = \int_0^{E_F} E \cdot g(E) \, dE
\end{equation}

where $g(E) = (V/(2\pi^2))(\sqrt{2m}/\hbar)^3 \sqrt{E}$ is the density of states. Evaluating the integral:
\begin{equation}
U = \frac{3}{5}NE_F = \frac{3}{5}N\frac{\hbar^2}{2m}(3\pi^2 n)^{2/3}
\end{equation}

The pressure is:
\begin{equation}
P = -\frac{\partial U}{\partial V}\bigg|_N = \frac{2}{5}n E_F = \frac{(3\pi^2)^{2/3}}{5}\frac{\hbar^2}{m}n^{5/3}
\end{equation}

Defining the degeneracy parameter $\theta = \kB T/E_F$, the equation of state is:
\begin{equation}
\boxed{P = \frac{(3\pi^2)^{2/3}}{5}\frac{\hbar^2}{m_e}n^{5/3} \quad (\text{non-relativistic, } T \ll E_F/\kB)}
\end{equation}

This is the equation of state for white dwarf stars, where electron degeneracy pressure supports the star against gravitational collapse.

\subsection{Relativistic Degenerate Electron Gas}

At ultra-high densities, the Fermi energy exceeds the electron rest mass energy $m_e c^2$, and relativistic corrections become necessary. The energy-momentum relation is:
\begin{equation}
E = \sqrt{(pc)^2 + (m_e c^2)^2}
\end{equation}

In the ultra-relativistic limit ($E_F \gg m_e c^2$), this reduces to $E \approx pc$. The total energy is:
\begin{equation}
U = \int_0^{p_F} pc \cdot g(p) \, dp
\end{equation}

where $g(p) = (V/\pi^2)(\hbar c)^{-3} p^2$ is the density of states in momentum space. Evaluating:
\begin{equation}
U = \frac{3}{4}Np_F c = \frac{3}{4}N\hbar c (3\pi^2 n)^{1/3}
\end{equation}

The pressure is:
\begin{equation}
P = -\frac{\partial U}{\partial V}\bigg|_N = \frac{1}{4}n p_F c = \frac{(3\pi^2)^{1/3}}{4}\hbar c \, n^{4/3}
\end{equation}

The equation of state is:
\begin{equation}
\boxed{P = \frac{(3\pi^2)^{1/3}}{4}\hbar c \, n^{4/3} \quad (\text{ultra-relativistic, } E_F \gg m_e c^2)}
\end{equation}

This is the equation of state for neutron stars, where relativistic electron (or neutron) degeneracy pressure supports the star.

\subsection{Chandrasekhar Limit}

The transition from non-relativistic to relativistic degeneracy has profound astrophysical consequences. For a white dwarf star, hydrostatic equilibrium requires:
\begin{equation}
\frac{dP}{dr} = -\rho(r) g(r)
\end{equation}

where $\rho(r)$ is the mass density and $g(r) = GM(r)/r^2$ is the gravitational acceleration.

For non-relativistic degeneracy ($P \propto n^{5/3} \propto \rho^{5/3}$), the pressure increases faster than gravity as density increases, and stable equilibrium exists for any mass.

For ultra-relativistic degeneracy ($P \propto n^{4/3} \propto \rho^{4/3}$), the pressure increases at the same rate as gravity. A critical mass exists beyond which no equilibrium is possible:
\begin{equation}
M_{\text{Ch}} = \frac{5.83}{\mu_e^2}M_\odot
\end{equation}

where $\mu_e$ is the mean molecular weight per electron and $M_\odot$ is the solar mass. For a carbon-oxygen white dwarf ($\mu_e \approx 2$), this yields $M_{\text{Ch}} \approx 1.4 M_\odot$.

This is the Chandrasekhar limit~\cite{chandrasekhar1931}: white dwarfs with $M > M_{\text{Ch}}$ collapse to neutron stars or black holes. The limit is a direct consequence of partition exclusion in relativistic regimes.

\subsection{Finite Temperature Corrections}

At finite temperature $T > 0$, thermal excitations partially populate states above $E_F$. The pressure is:
\begin{equation}
P = P_0(n) + P_{\text{thermal}}(n,T)
\end{equation}

where $P_0(n)$ is the zero-temperature degeneracy pressure and $P_{\text{thermal}}$ is the thermal correction.

For $T \ll E_F/\kB$ (strongly degenerate regime), the thermal correction is:
\begin{equation}
P_{\text{thermal}} = \frac{\pi^2}{3}n\kB T \left(\frac{\kB T}{E_F}\right) = \frac{\pi^2}{3}n(\kB T)^2/E_F
\end{equation}

The full equation of state is:
\begin{equation}
P = P_0(n)\left[1 + \frac{\pi^2}{3}\left(\frac{\kB T}{E_F}\right)^2 + \mathcal{O}(\theta^4)\right]
\end{equation}

where $\theta = \kB T/E_F$ is the degeneracy parameter.

\subsection{Partition Interpretation}

Degenerate matter exhibits pressure even at $T = 0$ because partition exclusion forces particles into high-energy states. This is purely a consequence of categorical constraints on partition occupancy (Pauli principle).

The $n^{5/3}$ scaling (non-relativistic) and $n^{4/3}$ scaling (ultra-relativistic) reflect the dimensionality of partition space. In three spatial dimensions, partition depth scales as $n \sim V^{-1/3}$, and energy scales as $E \sim n^2$ (non-relativistic) or $E \sim n$ (relativistic). The pressure $P \sim \partial E/\partial V$ then scales as $P \sim n^{5/3}$ or $P \sim n^{4/3}$, respectively.

\subsection{Experimental Validation}

The non-relativistic degenerate equation of state has been validated in white dwarf observations. Mass-radius relations for white dwarfs follow the predicted $M \propto R^{-3}$ scaling (from $P \propto n^{5/3}$) to within $\pm 5\%$~\cite{koester2009}.

The Chandrasekhar limit has been confirmed through observations of Type Ia supernovae, which occur when white dwarfs accrete mass and exceed $M_{\text{Ch}} \approx 1.4 M_\odot$. The narrow distribution of supernova luminosities (dispersion $\sigma_M \approx 0.1 M_\odot$) confirms the theoretical prediction~\cite{hillebrandt2000}.

The ultra-relativistic equation of state has been validated in neutron star observations. Measured masses cluster near $1.4 M_\odot$ with maximum observed mass $M_{\max} \approx 2.0 M_\odot$~\cite{demorest2010}, consistent with relativistic degeneracy pressure predictions.

Laboratory validation has been achieved in ultracold Fermi gases~\cite{ketterle2008}, where the equation of state $P(n,T)$ has been measured across the BCS-BEC crossover. In the strongly degenerate regime ($T/T_F \ll 1$), measured pressure agrees with the partition-based prediction $P = (2/5)nE_F$ to within $\pm 3\%$.

