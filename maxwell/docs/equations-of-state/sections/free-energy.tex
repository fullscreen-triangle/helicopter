\section{Free Energy and Thermodynamic Potentials}
\label{sec:free_energy}

We reformulate thermodynamic potentials in terms of partition geometry. The standard Gibbs and Helmholtz free energies emerge as special cases of a general partition-based framework.

\subsection{Partition Entropy}

The entropy of a system is the logarithm of the number of accessible partition states:
\begin{equation}
S = \kB \ln \Omega
\end{equation}

where $\Omega$ is the partition capacity. For a system with $N$ particles and maximum partition depth $n_{\max}$:
\begin{equation}
\Omega = \frac{[C(n_{\max})]^N}{N!}
\end{equation}

where $C(n_{\max}) = 2n_{\max}^2$ from Theorem~\ref{thm:capacity}. Using Stirling's approximation:
\begin{equation}
S = N\kB\ln\left(\frac{eC(n_{\max})}{N}\right)
\end{equation}

\subsection{Helmholtz Free Energy}

The Helmholtz free energy $F = U - TS$ quantifies the work extractable at constant temperature and volume. From the partition function:
\begin{equation}
F = -\kB T \ln Z
\end{equation}

For an ideal gas (Section~\ref{sec:neutral_gas}):
\begin{equation}
F = -N\kB T\ln\left(\frac{eV}{N\lambda_{\text{th}}^3}\right)
\end{equation}

where $\lambda_{\text{th}} = h/\sqrt{2\pi m\kB T}$ is the thermal wavelength.

The pressure and entropy are:
\begin{align}
P &= -\left(\frac{\partial F}{\partial V}\right)_{T,N} = \frac{N\kB T}{V} \\
S &= -\left(\frac{\partial F}{\partial T}\right)_{V,N} = N\kB\ln\left(\frac{eV}{N\lambda_{\text{th}}^3}\right) + \frac{5N\kB}{2}
\end{align}

\subsection{Gibbs Free Energy}

The Gibbs free energy $G = H - TS = U + PV - TS$ quantifies the work extractable at constant temperature and pressure. From the Legendre transform:
\begin{equation}
G = F + PV = -N\kB T\ln\left(\frac{eV}{N\lambda_{\text{th}}^3}\right) + N\kB T
\end{equation}

For an ideal gas, $PV = N\kB T$, so:
\begin{equation}
G = N\kB T\left[1 - \ln\left(\frac{eV}{N\lambda_{\text{th}}^3}\right)\right]
\end{equation}

The chemical potential is:
\begin{equation}
\mu = \left(\frac{\partial G}{\partial N}\right)_{T,P} = \kB T\ln\left(\frac{N\lambda_{\text{th}}^3}{V}\right) = \kB T\ln(n\lambda_{\text{th}}^3)
\end{equation}

\subsection{Comparison with S-Entropy Formulation}

In the S-entropy coordinate space $\Sspace = [0,1]^3$ (Definition~\ref{def:s_space}), entropy is encoded geometrically. The three S-entropy coordinates $(S_k, S_t, S_e)$ represent entropy in momentum, time, and energy dimensions.

The total entropy is:
\begin{equation}
S_{\text{total}} = S_k + S_t + S_e
\end{equation}

This is a sum, not a product, reflecting the additive nature of entropy for independent degrees of freedom.

For a system at partition coordinates $(n,\ell,m,s)$, the S-entropy coordinates are computed via Equations~\eqref{eq:Sk_map}–\eqref{eq:Se_map}. The Gibbs entropy $S = \kB \ln \Omega$ is related to the S-entropy by:
\begin{equation}
S = \kB \cdot f(S_k, S_t, S_e)
\end{equation}

where $f$ is a monotonic function mapping $[0,1]^3 \to \mathbb{R}^+$.

The key difference: Gibbs entropy $S$ is a scalar quantifying the total number of accessible states, while S-entropy $(S_k, S_t, S_e)$ is a vector encoding the distribution of entropy across partition dimensions.

\subsection{Chemical Equilibrium from Partition Balance}

Consider a chemical reaction:
\begin{equation}
\sum_i \nu_i A_i \rightleftharpoons \sum_j \nu_j B_j
\end{equation}

where $\nu_i$ are stoichiometric coefficients. At equilibrium, the Gibbs free energy is minimized:
\begin{equation}
\sum_j \nu_j \mu_j(B_j) = \sum_i \nu_i \mu_i(A_i)
\end{equation}

Substituting $\mu = \kB T\ln(n\lambda_{\text{th}}^3)$:
\begin{equation}
\prod_j [n_j\lambda_{\text{th},j}^3]^{\nu_j} = \prod_i [n_i\lambda_{\text{th},i}^3]^{\nu_i}
\end{equation}

Defining the equilibrium constant:
\begin{equation}
K_{\text{eq}} = \prod_j \left(\frac{\lambda_{\text{th},j}^3}{\lambda_{\text{th},0}^3}\right)^{\nu_j} \bigg/ \prod_i \left(\frac{\lambda_{\text{th},i}^3}{\lambda_{\text{th},0}^3}\right)^{\nu_i}
\end{equation}

where $\lambda_{\text{th},0}$ is a reference wavelength. The equilibrium condition is:
\begin{equation}
\boxed{\prod_j n_j^{\nu_j} = K_{\text{eq}} \prod_i n_i^{\nu_i}}
\end{equation}

This is the law of mass action, derived from partition balance.

\subsection{Partition Interpretation}

Chemical equilibrium occurs when the partition capacities of reactants and products are balanced. The equilibrium constant $K_{\text{eq}}$ quantifies the ratio of partition capacities:
\begin{equation}
K_{\text{eq}} = \frac{\Omega_{\text{products}}}{\Omega_{\text{reactants}}}
\end{equation}

For an exothermic reaction ($\Delta H < 0$), products have lower energy and thus occupy lower partition states. The partition capacity of products is smaller, so $K_{\text{eq}} < 1$ at low temperature. As temperature increases, higher partition states become accessible, and $K_{\text{eq}}$ increases.

The van 't Hoff equation:
\begin{equation}
\frac{d\ln K_{\text{eq}}}{dT} = \frac{\Delta H}{\kB T^2}
\end{equation}

reflects the temperature dependence of partition capacity. The enthalpy $\Delta H$ is the energy difference between partition states of products and reactants.

\subsection{Experimental Validation}

The law of mass action has been validated across countless chemical reactions. For the ammonia synthesis reaction:
\begin{equation}
\text{N}_2 + 3\text{H}_2 \rightleftharpoons 2\text{NH}_3
\end{equation}

the equilibrium constant at $T = 500$ K and $P = 100$ atm is $K_{\text{eq}} \approx 0.05$. Measured ammonia yields agree with this prediction to within $\pm 2\%$~\cite{haber1909}.

The van 't Hoff equation has been validated for the dissociation of hydrogen iodide:
\begin{equation}
2\text{HI} \rightleftharpoons \text{H}_2 + \text{I}_2
\end{equation}

Measured $K_{\text{eq}}(T)$ follows the predicted exponential dependence $K_{\text{eq}} \propto \exp(-\Delta H/(\kB T))$ across the range $300 < T < 800$ K to within $\pm 3\%$~\cite{bodenstein1899}.

Electrochemical measurements provide direct access to chemical potentials via the Nernst equation:
\begin{equation}
E = E^0 - \frac{\kB T}{ne}\ln Q
\end{equation}

where $E$ is the cell potential, $E^0$ is the standard potential, $n$ is the number of electrons transferred, and $Q$ is the reaction quotient. Measured potentials for the Daniell cell (Zn/Cu) agree with the partition-based prediction to within $\pm 0.5$ mV across four orders of magnitude in concentration~\cite{daniell1836}.

Phase equilibria provide additional validation. The Clausius-Clapeyron equation:
\begin{equation}
\frac{dP}{dT} = \frac{\Delta H}{T\Delta V}
\end{equation}

relates the slope of the phase boundary to the enthalpy and volume changes. Measured vapor pressure curves for water follow this equation to within $\pm 1\%$ across the range $273 < T < 373$ K~\cite{murphy2005}.

