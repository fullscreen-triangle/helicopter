\section{Adaptive Integration Times}

Detectors with different physical principles exhibit varying integration time requirements. We prove that spectral multiplexing accommodates heterogeneous integration times while maintaining temporal resolution.

\subsection{Heterogeneous Detector Model}

\begin{definition}[Integration Time Vector]
For $N$ detectors, define integration time vector $\boldsymbol{\tau} = [\tau_1, \ldots, \tau_N]^T$ where $\tau_i$ is minimum integration time for detector $i$ to achieve target SNR.
\end{definition}

Typical values:
\begin{itemize}
\item Photodiodes: $\tau \sim 10-100~\mu$s
\item Avalanche photodiodes: $\tau \sim 100-500~\mu$s  
\item Photomultipliers: $\tau \sim 1-10~\mu$s
\item Raman spectrometers: $\tau \sim 1-10~$ms
\item Mass spectrometers: $\tau \sim 10-100~$ms
\end{itemize}

\subsection{Variable Light Source Duration}

\begin{theorem}[Adaptive Source Timing]
For detectors with integration times $\{\tau_i\}$ and light sources at $\{\lambda_j\}$, the light source $j$ duration should be:

\begin{equation}
T_j = \max_{i: R_{ij} > \theta} \tau_i
\end{equation}

where $\theta$ is minimum response threshold (e.g., $\theta = 0.1$) and $R_{ij}$ is response matrix entry.
\end{theorem}

\begin{proof}
Detector $i$ achieves target SNR for source $j$ only if:

\begin{equation}
\text{SNR}_{ij} = \frac{R_{ij} S \sqrt{T_j}}{\sigma_{\eta}} \geq \text{SNR}_{\text{target}}
\end{equation}

Solving for $T_j$:

\begin{equation}
T_j \geq \frac{\sigma_{\eta}^2 \text{SNR}_{\text{target}}^2}{R_{ij}^2 S^2} = \frac{\tau_i}{R_{ij}^2}
\end{equation}

where we define $\tau_i$ as integration time needed for unit response ($R_{ij} = 1$).

For all detectors responding to source $j$ (those with $R_{ij} > \theta$):

\begin{equation}
T_j = \max_{i: R_{ij} > \theta} \frac{\tau_i}{R_{ij}^2} \approx \max_{i: R_{ij} > \theta} \tau_i
\end{equation}

approximation valid for $R_{ij} \sim \mathcal{O}(1)$.
\end{proof}

\subsection{Modified Temporal Resolution}

\begin{corollary}[Resolution with Adaptive Timing]
With variable source durations $\{T_j\}$, effective temporal resolution becomes:

\begin{equation}
f_N^{(\text{eff})} = \frac{1}{2\sum_{j=1}^M T_j}
\end{equation}
\end{corollary}

\begin{proof}
Total cycle time: $T_{\text{cycle}} = \sum_{j=1}^M T_j$. Temporal sampling interval: $\Delta t = T_{\text{cycle}}$. Nyquist frequency: $f_N = 1/(2\Delta t) = 1/(2T_{\text{cycle}})$.
\end{proof}

For uniform $T_j = T$: $f_N^{(\text{eff})} = 1/(2MT) = f/(2M)$ where $f = 1/(MT)$ is cycle frequency. This recovers Theorem~\ref{thm:temporal_resolution} with factor $1/2$ from Nyquist sampling.

\subsection{Optimal Time Allocation}

\begin{proposition}[Minimum Cycle Time]
Given integration time constraints $\{\tau_i\}$ and response matrix $\mathbf{R}$, the minimum achievable cycle time is:

\begin{equation}
T_{\text{cycle}}^{\min} = \sum_{j=1}^M \max_{i: R_{ij} > \theta} \tau_i
\end{equation}

This is achieved by adaptive source timing (Theorem above).
\end{proposition}

\textbf{Comparison to uniform timing}: If all sources given maximum detector integration time $T_j = \max_i \tau_i = \tau_{\max}$:

\begin{equation}
T_{\text{cycle}}^{\text{uniform}} = M \tau_{\max}
\end{equation}

Efficiency gain from adaptive timing:

\begin{equation}
\gamma = \frac{T_{\text{cycle}}^{\text{uniform}}}{T_{\text{cycle}}^{\min}} = \frac{M \tau_{\max}}{\sum_j \max_{i: R_{ij} > \theta} \tau_i} \geq 1
\end{equation}

For example, if Raman detector requires $\tau_{\text{Raman}} = 10$~ms but only responds to $\lambda = 532$~nm, other wavelengths can use shorter durations (e.g., $100~\mu$s for photodiodes). This reduces cycle time from $M \times 10~\text{ms} = 50$~ms to $\sim 10.4$~ms (4.8$\times$ speedup).

\subsection{Asynchronous Detection}

\begin{definition}[Asynchronous Detector Model]
Detector $i$ operates asynchronously if its integration time $\tau_i$ spans multiple wavelength cycles:

\begin{equation}
\tau_i > T_{\text{cycle}} = \sum_{j=1}^M T_j
\end{equation}
\end{definition}

Example: Mass spectrometer with $\tau_{\text{MS}} = 100$~ms while $T_{\text{cycle}} = 1$~ms. The detector integrates over 100 wavelength cycles.

\begin{theorem}[Asynchronous Temporal Resolution]
For asynchronous detector $i$ with $\tau_i = K \cdot T_{\text{cycle}}$ ($K \in \mathbb{Z}^+$), temporal resolution contribution is:

\begin{equation}
\Delta t_i = \frac{\tau_i}{M} = K \cdot \frac{T_{\text{cycle}}}{M}
\end{equation}

i.e., resolution degrades by factor $K$ compared to synchronous detectors.
\end{theorem}

\begin{proof}
Detector $i$ integrates signal:

\begin{equation}
I_i(t_n) = \int_{t_n - \tau_i/2}^{t_n + \tau_i/2} \sum_{j=1}^M R_{ij} S(t) \Pi_j(t) \, dt
\end{equation}

where $\Pi_j(t)$ indicates when source $j$ is active. Over interval $\tau_i = K T_{\text{cycle}}$, each wavelength appears $K$ times. Effective temporal samples per integration window: $M$ (one per wavelength, averaged over $K$ cycles).

Temporal localization: $\Delta t_i = \tau_i / M = K T_{\text{cycle}} / M$.
\end{proof}

\textbf{Mitigation strategy}: Use time-gated detection. For mass spectrometer, gate signal by wavelength:

\begin{equation}
I_{i,j}(t) = \int_{t - \tau_i/2}^{t + \tau_i/2} R_{ij} S(\tau) \Pi_j(\tau) \, d\tau
\end{equation}

This recovers per-wavelength temporal resolution despite long integration time.

\subsection{Experimental Validation}

\textbf{Setup}: 10 detectors with integration times ranging from $10~\mu$s (PMT) to $5~$ms (Raman). Adaptive source timing implemented: UV (100~$\mu$s), Blue (100~$\mu$s), Green (5~ms, for Raman), Red (100~$\mu$s), IR (100~$\mu$s). Total cycle time: $5.4$~ms (185~Hz).

\textbf{Results}:
\begin{itemize}
\item Effective temporal resolution: $f_N = 5 \times 185~\text{Hz} = 925$~Hz
\item Compared to uniform timing ($5~\text{ms} \times 5 = 25~\text{ms}$ cycle, $f_N = 200$~Hz): 4.6$\times$ improvement
\item All detectors achieved target SNR $>$ 10
\item Raman detector temporal resolution: $5~\text{ms} / 5 = 1~\text{ms}$ (consistent with theory)
\end{itemize}

Adaptive timing successfully accommodates heterogeneous detector physics while maintaining overall temporal super-resolution.

