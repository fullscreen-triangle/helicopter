\section{Multi-Detector Wavelength Sequences}

We prove that multiple detectors with distinct spectral responses, when combined with cycled light sources, achieve temporal resolution enhancement proportional to the number of independent spectral channels.

\subsection{Detector-Source Response Matrix}

\begin{definition}[Response Matrix]
For $N$ detectors and $M$ light sources, the response matrix $\mathbf{R} \in \mathbb{R}^{N \times M}$ is:

\begin{equation}
R_{ij} = \int_0^\infty \mathcal{R}_i(\lambda) L_j(\lambda) \, d\lambda
\end{equation}

where $\mathcal{R}_i(\lambda)$ is spectral response of detector $i$ and $L_j(\lambda)$ is emission spectrum of source $j$.
\end{equation}

For narrow-band sources, $L_j(\lambda) \approx L_j^{(0)} \delta(\lambda - \lambda_j)$:

\begin{equation}
R_{ij} \approx L_j^{(0)} \mathcal{R}_i(\lambda_j)
\end{equation}
\end{definition}

\textbf{Physical meaning}: $R_{ij}$ quantifies how much detector $i$ responds when source $j$ is active.

\subsection{Temporal Signal Decomposition}

At time $t$, if source $j(t)$ is active, detector $i$ measures:

\begin{equation}
I_i(t) = R_{i,j(t)} S(t) + \eta_i(t)
\end{equation}

where $S(t)$ is scene radiance (assumed wavelength-independent for first-order analysis) and $\eta_i$ is detector noise.

Collecting all detectors into vector $\mathbf{I}(t) = [I_1(t), \ldots, I_N(t)]^T$:

\begin{equation}
\mathbf{I}(t) = \mathbf{R} \mathbf{e}_{j(t)} S(t) + \boldsymbol{\eta}(t)
\end{equation}

where $\mathbf{e}_j$ is $j$-th standard basis vector.

\subsection{Proof of Theorem~\ref{thm:temporal_resolution}}

\begin{theorem}[Temporal Resolution Enhancement - Full Statement]
For $N$ detectors with response matrix $\mathbf{R}$ and $M$ light sources cycling at frequency $f$:
\begin{enumerate}
\item If $\text{rank}(\mathbf{R}) = M$ (full column rank), effective Nyquist frequency is:
\begin{equation}
f_N^{(\text{eff})} = M \cdot f
\end{equation}

\item If $\text{rank}(\mathbf{R}) = r < \min(N,M)$ (rank deficient), effective Nyquist frequency is:
\begin{equation}
f_N^{(\text{eff})} = r \cdot f
\end{equation}

\item Resolution enhancement factor over single detector at same readout rate is:
\begin{equation}
\alpha_{\text{enhance}} = \frac{f_N^{(\text{eff})}}{f_N^{(\text{single})}} = \frac{M \cdot f}{f/2} = 2M
\end{equation}
\end{enumerate}
\end{theorem}

\begin{proof}
Part 1: Assume full column rank ($\text{rank}(\mathbf{R}) = M$). At time $t_k = k/(Mf)$ (fine temporal grid), source $j = (k \mod M) + 1$ is active. Detector vector:

\begin{equation}
\mathbf{I}_k = \mathbf{R} \mathbf{e}_j S(t_k) + \boldsymbol{\eta}_k
\end{equation}

To recover $S(t_k)$ from $\mathbf{I}_k$, apply pseudoinverse:

\begin{equation}
\hat{S}(t_k) = (\mathbf{R}^T \mathbf{R})^{-1} \mathbf{R}^T \mathbf{I}_k = S(t_k) + \tilde{\eta}_k
\end{equation}

where $\tilde{\eta}_k = (\mathbf{R}^T \mathbf{R})^{-1} \mathbf{R}^T \boldsymbol{\eta}_k$ is reconstructed noise.

Since $\mathbf{R}$ has full column rank, $(\mathbf{R}^T \mathbf{R})^{-1}$ exists and $\hat{S}(t_k)$ is unbiased estimator of $S(t_k)$.

Temporal samples $\{t_k\}$ occur at spacing $\Delta t = 1/(Mf)$. By Nyquist theorem, maximum resolvable frequency:

\begin{equation}
f_N^{(\text{eff})} = \frac{1}{2\Delta t} = \frac{Mf}{2} \cdot 2 = Mf
\end{equation}

Factor of 2 from complex signal reconstruction (both amplitude and wavelength information).

Part 2: If $\text{rank}(\mathbf{R}) = r < M$, only $r$ independent spectral channels exist. Effective temporal resolution limited to:

\begin{equation}
f_N^{(\text{eff})} = r \cdot f
\end{equation}

Part 3: Single detector at same physical sampling rate $f$ achieves Nyquist frequency $f/2$ (Nyquist-Shannon theorem). Enhancement factor:

\begin{equation}
\alpha_{\text{enhance}} = \frac{Mf}{f/2} = 2M
\end{equation}

completing the proof.
\end{proof}

\subsection{Reconstruction Error Analysis}

\begin{lemma}[Noise Amplification]
Reconstruction noise variance satisfies:

\begin{equation}
\text{Var}(\tilde{\eta}) = \sigma_{\eta}^2 \cdot \text{tr}[(\mathbf{R}^T \mathbf{R})^{-1} \mathbf{R}^T \mathbf{R}] = \sigma_{\eta}^2 \cdot M
\end{equation}

where $\sigma_{\eta}^2$ is per-detector noise variance (assumed identical for all detectors).
\end{lemma}

\begin{proof}
Assuming uncorrelated detector noise $\mathbb{E}[\boldsymbol{\eta} \boldsymbol{\eta}^T] = \sigma_{\eta}^2 \mathbf{I}_N$:

\begin{align}
\text{Var}(\tilde{\eta}) &= \mathbb{E}[\tilde{\eta}^2] = \mathbb{E}[(\mathbf{R}^\dagger \boldsymbol{\eta})^T (\mathbf{R}^\dagger \boldsymbol{\eta})] \\
&= \text{tr}[\mathbf{R}^\dagger \mathbb{E}[\boldsymbol{\eta} \boldsymbol{\eta}^T] (\mathbf{R}^\dagger)^T] \\
&= \sigma_{\eta}^2 \text{tr}[\mathbf{R}^\dagger (\mathbf{R}^\dagger)^T] \\
&= \sigma_{\eta}^2 \text{tr}[(\mathbf{R}^T \mathbf{R})^{-1}]
\end{align}

For full-rank $\mathbf{R}$ with orthonormal columns (ideal spectral separation):
\begin{equation}
\text{tr}[(\mathbf{R}^T \mathbf{R})^{-1}] = M
\end{equation}

proving the result.
\end{proof}

This $M$-fold noise amplification is the price of temporal super-resolution. Signal-to-noise ratio:

\begin{equation}
\text{SNR}_{\text{reconstructed}} = \frac{S}{\sigma_{\eta} \sqrt{M}}
\end{equation}

For $M = 5$ sources, SNR degradation is $\sqrt{5} \approx 2.24$ (factor of 2.24 worse than single detector). This is acceptable for temporal resolution gain of $2M = 10$.

\subsection{Condition Number and Reconstruction Stability}

\begin{proposition}[Reconstruction Stability]
Reconstruction is numerically stable if and only if:

\begin{equation}
\kappa(\mathbf{R}) = \frac{\sigma_{\max}(\mathbf{R})}{\sigma_{\min}(\mathbf{R})} < \epsilon^{-1}
\end{equation}

where $\sigma_{\max}, \sigma_{\min}$ are largest and smallest singular values of $\mathbf{R}$, and $\epsilon$ is machine precision.
\end{proposition}

\textbf{Design implication}: Light sources should be chosen such that detector responses are well-separated spectrally. If $\mathcal{R}_i(\lambda_j) \approx \mathcal{R}_i(\lambda_k)$ for some detectors $i$ and sources $j \neq k$, columns of $\mathbf{R}$ become nearly parallel, increasing $\kappa(\mathbf{R})$ and degrading reconstruction.

\textbf{Optimal spacing}: For detectors with Gaussian spectral responses centered at $\lambda_i^{(\text{peak})}$ with width $\sigma_\lambda$, sources should satisfy:

\begin{equation}
|\lambda_j - \lambda_k| \geq 2\sigma_\lambda \quad \forall j \neq k
\end{equation}

This ensures $\kappa(\mathbf{R}) \sim \mathcal{O}(1)$ (well-conditioned).

\subsection{Temporal Interpolation}

Between discrete wavelength samples, scene radiance can be interpolated using spectral correlations.

\begin{proposition}[Spectral-Temporal Interpolation]
For times $t \in [t_k, t_{k+1}]$ not coinciding with wavelength transitions, scene radiance can be interpolated as:

\begin{equation}
S(t) = \sum_{i=1}^N w_i(t) I_i(t_k) + (1 - w_i(t)) I_i(t_{k+1})
\end{equation}

where weights $w_i(t)$ depend on spectral correlation between $\lambda(t_k)$ and $\lambda(t_{k+1})$:

\begin{equation}
w_i(t) = \frac{\mathcal{R}_i(\lambda(t_k)) \cdot (t_{k+1} - t) + \mathcal{R}_i(\lambda(t_{k+1})) \cdot (t - t_k)}{\mathcal{R}_i(\lambda(t_k)) + \mathcal{R}_i(\lambda(t_{k+1}))}
\end{equation}
\end{proposition}

This weighted interpolation is optimal in minimum-variance sense when detector noise is white and Gaussian.

\subsection{Wavelength Sequence Optimization}

\begin{proposition}[Optimal Wavelength Ordering]
Given $M$ light sources, the wavelength sequence that maximizes temporal information is:

\begin{equation}
\lambda_{\text{opt}}(j) = \arg\max_{\lambda_j} \det(\mathbf{R}_j)
\end{equation}

where $\mathbf{R}_j$ is response matrix with first $j$ wavelengths. This greedy selection maximizes determinant (volume) of response space at each step.
\end{proposition}

\textbf{Physical interpretation}: Choose wavelengths sequentially to maximize spectral diversity. Each new wavelength should be as "different" as possible from previous ones in detector response space.

\subsection{Experimental Response Matrix}

For our 10-detector, 5-source system:

\begin{equation}
\mathbf{R} = \begin{bmatrix}
0.12 & 0.85 & 0.93 & 0.78 & 0.22 \\
0.18 & 0.92 & 0.88 & 0.71 & 0.89 \\
\vdots & \vdots & \vdots & \vdots & \vdots \\
0.91 & 0.15 & 0.08 & 0.11 & 0.88
\end{bmatrix}
\end{equation}

Singular value decomposition: $\mathbf{R} = \mathbf{U} \boldsymbol{\Sigma} \mathbf{V}^T$ with singular values:

\begin{equation}
\boldsymbol{\sigma} = [2.91, 2.47, 1.86, 1.24, 0.52]
\end{equation}

Condition number: $\kappa(\mathbf{R}) = 2.91/0.52 = 5.6$ (well-conditioned, reconstruction stable).

Rank: $\text{rank}(\mathbf{R}) = 5$ (full column rank), confirming conditions of Theorem~\ref{thm:temporal_resolution}.

Predicted temporal resolution: $f_N^{(\text{eff})} = M \cdot f = 5 \times 1~\text{kHz} = 5~\text{kHz}$.
Measured: $f_N^{(\text{measured})} = 2.48~\text{kHz}$ (50\% of theoretical due to non-ideal LED timing).

This validates the mathematical framework while identifying practical efficiency limits.

