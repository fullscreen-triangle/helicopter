\section{Pixel Maxwell Demon: Categorical Observer at Spatial Locations}

\subsection{Definition and Structure}

A pixel Maxwell demon (PMD) is a categorical observer positioned at spatial location $\mathbf{r} \in \mathbb{R}^3$ that accesses S-entropy coordinates $(S_k, S_t, S_e)$ through molecular demon lattice aggregation and virtual detector instantiation.

\begin{definition}[Pixel Maxwell Demon]
\label{def:pixel_maxwell_demon}
A pixel Maxwell demon is a five-tuple:
\begin{equation}
\text{PMD} = (\mathbf{r}, \mathcal{M}, \mathcal{D}, \mathcal{H}, \mathbf{S})
\end{equation}
where:
\begin{itemize}
\item $\mathbf{r} \in \mathbb{R}^3$: spatial position
\item $\mathcal{M} = \{M_1, M_2, \ldots, M_m\}$: molecular demon lattice (one demon per molecule type)
\item $\mathcal{D} = \{D_1, D_2, \ldots, D_d\}$: virtual detector set
\item $\mathcal{H} = \{H_1, H_2, \ldots, H_h\}$: hypothesis space about pixel content
\item $\mathbf{S} = (S_k, S_t, S_e)$: categorical state coordinates
\end{itemize}
\end{definition}

\subsection{Molecular Demon Lattice}

Each molecular species at position $\mathbf{r}$ has an associated molecular demon $M_i$ that aggregates ensemble properties for that species.

\begin{definition}[Molecular Demon]
\label{def:molecular_demon}
For molecule type $i$, the molecular demon $M_i$ maintains:
\begin{align}
n_i(\mathbf{r}) &: \text{number density (molecules/m³)} \\
f_i &: \text{vibrational frequency (Hz)} \\
\phi_i &: \text{oscillator phase (radians)} \\
m_i &: \text{molecular mass (kg)} \\
\sigma_i &: \text{collision cross-section (m²)}
\end{align}
\end{definition}

For atmospheric conditions at temperature $T = 288$ K and pressure $P = 101.325$ kPa, molecular demons track the primary atmospheric constituents:

\begin{center}
\begin{tabular}{|l|c|c|c|c|}
\hline
\textbf{Species} & $f_i$ (Hz) & $n_i$ (m$^{-3}$) & $m_i$ (kg) & $\sigma_i$ (m²) \\
\hline
O$_2$ & $4.74 \times 10^{13}$ & $5.39 \times 10^{24}$ & $5.31 \times 10^{-26}$ & $3.47 \times 10^{-19}$ \\
N$_2$ & $6.99 \times 10^{13}$ & $2.04 \times 10^{25}$ & $4.65 \times 10^{-26}$ & $3.74 \times 10^{-19}$ \\
H$_2$O & $1.10 \times 10^{14}$ & $3.64 \times 10^{23}$ & $2.99 \times 10^{-26}$ & $4.48 \times 10^{-19}$ \\
CO$_2$ & $4.03 \times 10^{13}$ & $1.04 \times 10^{22}$ & $7.31 \times 10^{-26}$ & $4.20 \times 10^{-19}$ \\
Ar & $9.70 \times 10^{12}$ & $2.38 \times 10^{23}$ & $6.63 \times 10^{-26}$ & $3.64 \times 10^{-19}$ \\
\hline
\end{tabular}
\end{center}

The molecular demon lattice $\mathcal{M} = \{M_{\text{O}_2}, M_{\text{N}_2}, M_{\text{H}_2\text{O}}, M_{\text{CO}_2}, M_{\text{Ar}}\}$ aggregates over $N_{\text{total}} \sim 2.6 \times 10^{25}$ molecules/m³ through $m = 5$ species representatives, reducing query complexity from $\mathcal{O}(N_{\text{total}})$ to $\mathcal{O}(m) \approx \mathcal{O}(1)$.

\subsection{S-Entropy Coordinate Computation}

The categorical state $\mathbf{S} = (S_k, S_t, S_e)$ at position $\mathbf{r}$ is computed from molecular demon lattice properties.

\subsubsection{Knowledge Entropy from Information Density}

The knowledge entropy $S_k$ quantifies information deficit through molecular vibrational frequency spectrum:
\begin{equation}
S_k(\mathbf{r}) = 1 - \exp\left(-\frac{\rho(\mathbf{r})}{\rho_{\text{ref}}}\right)
\end{equation}
where information density is:
\begin{equation}
\rho(\mathbf{r}) = \sum_{i \in \mathcal{M}} n_i(\mathbf{r}) \log_2\left(\frac{f_i}{f_{\text{ref}}}\right)
\end{equation}

The reference frequency $f_{\text{ref}} = 1$ Hz provides dimensional consistency, with the logarithm converting frequency ratios to information bits. High vibrational frequencies (e.g., H$_2$O at $1.10 \times 10^{14}$ Hz) contribute more information per molecule than low frequencies (e.g., Ar at $9.70 \times 10^{12}$ Hz), reflecting greater categorical richness in molecular electronic structure.

\subsubsection{Temporal Entropy from Phase Coherence}

The temporal entropy $S_t$ measures phase coherence of molecular oscillators at position $\mathbf{r}$:
\begin{equation}
S_t(\mathbf{r}) = 1 - \left|\frac{1}{|\mathcal{M}|} \sum_{i \in \mathcal{M}} e^{i\phi_i}\right|
\end{equation}

For perfect phase synchronization ($\phi_i = \phi_0$ for all $i$), $S_t = 0$. For complete decoherence with uniformly distributed phases, $S_t \to 1$. Atmospheric molecules exhibit partial phase coherence due to collision-induced phase coupling and long-range Van der Waals interactions, typically yielding $S_t \in [0.3, 0.7]$ under standard conditions.

\subsubsection{Evolutionary Entropy from Frequency Variance}

The evolutionary entropy $S_e$ quantifies the heterogeneity in the vibrational frequency distribution:
\begin{equation}
S_e(\mathbf{r}) = \sqrt{\frac{\text{Var}(\{f_i : i \in \mathcal{M}\})}{\langle f \rangle^2}}
\end{equation}

This is the coefficient of variation of vibrational frequencies. High $S_e$ indicates diverse molecular species with widely varying frequencies (high evolutionary potential); low $S_e$ indicates homogeneous molecular composition (low evolutionary potential).

\begin{figure*}[htbp]
\centering
\includegraphics[width=0.95\textwidth]{figures/multi_modal_detector_analysis.png}
\caption{\textbf{Signal processing validation across five computational imaging modalities.} 
Comprehensive analysis demonstrates consistency of categorical coordinate extraction across fluorescence, phase contrast, wavelength-specific, and darkfield imaging modes.
Each row represents one imaging modality with standardized analysis pipeline: \textbf{(Column 1)} Original image, \textbf{(Column 2)} Phase map via Hilbert transform, \textbf{(Column 3)} 2D power spectrum (log scale), \textbf{(Column 4)} Edge detection (Canny), \textbf{(Column 5)} Circular phase histogram, \textbf{(Column 6)} Gradient magnitude, \textbf{(Column 7)} 2D autocorrelation (central region), \textbf{(Column 8)} Frequency band distribution.
\textbf{Row A: Fluorescence (561nm).} Original image shows cytoskeletal structures with rhodamine labeling (excitation 561nm, emission 580nm). Phase map reveals $\pm\pi$ phase structure with mean phase $= -0.0012 \pm 1.713$~rad. Power spectrum shows dominant low-frequency component (97.5\% power in low band) indicating smooth spatial structure. Circular phase histogram shows uniform angular distribution, indicating isotropic phase structure. 
\textbf{Row B: Phase contrast.} Original image shows cellular structures via phase-contrast microscopy (DIC-like contrast). Phase map reveals complex phase structure with mean $= -0.0318 \pm 1.899$~rad, higher variance than fluorescence indicating richer phase information. Power spectrum shows 99.9\% power in low band, indicating very smooth phase gradients. Circular phase histogram shows slight anisotropy (preferential alignment along 45° and 135°). Gradient magnitude mean $= 0.024$. 
\textbf{Row C: 450nm (blue wavelength).} Original image shows same cellular structures imaged at 450nm (blue light). Phase map, power spectrum, and gradient structure nearly identical to 561nm fluorescence (phase mean $= -0.0012 \pm 1.713$~rad, gradient mean $= 0.024$), demonstrating wavelength-independent categorical structure. Frequency band distribution identical (97.5\% low, 0.9\% mid, 1.6\% high), confirming categorical coordinates are wavelength-invariant.
\textbf{Row D: 650nm (red wavelength).} Original image at 650nm shows identical structure to 450nm and 561nm. Phase map, power spectrum, gradient magnitude, and all statistical measures identical to within measurement precision (phase mean $= -0.0012$~rad, gradient mean $= 0.024$, frequency bands 97.5\%/0.9\%/1.6\%), providing strong evidence that categorical structure is intrinsic to image content rather than wavelength-dependent.
\textbf{Row E: Darkfield.} Original image shows cellular structures via darkfield illumination (scattered light only, no direct illumination). Intensity mean $= 0.198 \pm 0.068$ (darker overall than brightfield/fluorescence). Phase map shows mean $= 0.013 \pm 1.713$~rad. 
\textbf{Cross-modality consistency:} Phase distributions across all five modalities show similar structure (uniform angular distribution, $\pm\pi$ range). Power spectra consistently show dominant low-frequency component (>97\% power), indicating categorical structure is predominantly smooth with localized sharp transitions. 
\textbf{Wavelength independence:} Three wavelength-specific modalities (450nm, 561nm, 650nm) show identical statistical properties to within measurement precision, demonstrating that categorical coordinates are wavelength-invariant. This supports the claim that categorical structure is intrinsic to image content rather than wavelength-dependent measurement artifact.
\textbf{Modality-specific variations:} Phase contrast shows highest phase variance (std $= 1.899$~rad) and longest correlation length ($\approx 100$~pixels), indicating phase-contrast imaging preserves more categorical information than intensity-based modalities.}
\label{fig:signal_processing_multimodal}
\end{figure*}

\subsection{Virtual Detector Array}

Virtual detectors enable hypothesis testing through categorical coordinate queries without the instantiation of physical apparatus. Each detector $D \in \mathcal{D}$ is a functional mapping molecular demon states and categorical coordinates to measurement predictions and consistency flags:
\begin{equation}
D: \mathcal{M} \times \mathbf{S} \to \mathbb{R} \times \{\text{consistent}, \text{inconsistent}\}
\end{equation}

\subsubsection{Virtual Thermometer}

Temperature is computed from mean squared molecular velocity via equipartition:
\begin{equation}
T = \frac{1}{3k_B} \sum_{i \in \mathcal{M}} n_i m_i \langle v_i^2 \rangle
\end{equation}
where $\langle v_i^2 \rangle = 3k_B T / m_i$ from kinetic theory. This is a consistency equation: the temperature computed from molecular velocities must match the temperature inferred from vibrational frequency distribution via $f_i \propto \sqrt{k_B T/\mu_i}$ for reduced mass $\mu_i$.

\subsubsection{Virtual Barometer}

Pressure follows from ideal gas law applied to each molecular species:
\begin{equation}
P = \sum_{i \in \mathcal{M}} n_i k_B T
\end{equation}

For atmospheric composition at $T = 288$ K, this yields $P \approx 101.3$ kPa, validating the accuracy of molecular demon lattice aggregation.

\subsubsection{Virtual Hygrometer}

Relative humidity is the ratio of the partial pressure of water vapour to the saturation vapour pressure:
\begin{equation}
\text{RH} = \frac{n_{\text{H}_2\text{O}} k_B T}{P_{\text{sat}}(T)} \times 100\%
\end{equation}
where $P_{\text{sat}}(T)$ is computed from the Antoine equation. This detector cross-validates atmospheric composition hypotheses against thermodynamic consistency constraints.

\subsubsection{Virtual IR Spectrometer}

Infrared absorption intensity at wavenumber $\nu$ is:
\begin{equation}
I_{\text{IR}}(\nu) = \sum_{i \in \mathcal{M}} n_i \sigma_i(\nu) \exp\left(-\frac{h\nu}{k_B T}\right)
\end{equation}
where $\sigma_i(\nu)$ is the absorption cross-section for species $i$ at wavenumber $\nu$. The exponential factor reflects Boltzmann population of vibrational states.

\subsubsection{Virtual Raman Spectrometer}

Raman scattering intensity at a frequency shift $\Delta\nu$ from the excitation frequency $\nu_0$ is:
\begin{equation}
I_{\text{Raman}}(\Delta\nu) = \sum_{i \in \mathcal{M}} n_i \alpha_i^2 (\nu_0 \pm \Delta\nu)^4
\end{equation}
where $\alpha_i$ is polarizability. The $\nu^4$ dependence produces stronger Raman scattering at higher frequencies (Stokes and anti-Stokes shifted).

\subsubsection{Virtual Mass Spectrometer}

Mass spectrum is a discrete function at molecular mass-to-charge ratios:
\begin{equation}
I_{\text{MS}}(m/z) = \sum_{i \in \mathcal{M}} n_i \delta\left(\frac{m}{z} - \frac{m_i}{z_i}\right)
\end{equation}

For singly ionised molecules ($z_i = 1$), this provides direct species identification through molecular mass.

\subsection{Consilience Engine for Hypothesis Validation}

Given hypothesis space $\mathcal{H} = \{H_1, H_2, \ldots, H_h\}$ about pixel content, the consilience engine cross-validates each hypothesis against all virtual detectors.

\begin{definition}[Consilience]
\label{def:consilience}
Hypothesis $H$ has consilience:
\begin{equation}
C(H) = \frac{1}{|\mathcal{D}|} \sum_{D \in \mathcal{D}} \mathbb{1}[\text{$D$ consistent with $H$}]
\end{equation}
where the indicator function $\mathbb{1}[\cdot]$ equals 1 if detector $D$ output is consistent with hypothesis $H$ predictions, and 0 otherwise.
\end{definition}

\begin{theorem}[Consilience Maximization]
\label{thm:consilience_maximization}
The hypothesis with maximum consilience is the most probable interpretation:
\begin{equation}
H^* = \arg\max_{H \in \mathcal{H}} C(H)
\end{equation}
\end{theorem}

\begin{proof}
Each virtual detector $D$ provides independent evidence. If detector $D$ has false positive probability $p_D < 0.5$ (more likely to correctly reject incorrect hypotheses than incorrectly accept them), the probability that all $|\mathcal{D}|$ detectors simultaneously give false positives for incorrect hypothesis $H_{\text{wrong}}$ is:
\begin{equation}
P(\text{all false positives}) = \prod_{D \in \mathcal{D}} p_D \leq p_{\max}^{|\mathcal{D}|}
\end{equation}

For $p_{\max} = 0.3$ (conservative) and $|\mathcal{D}| = 6$ detectors, $P(\text{all false}) \leq 0.3^6 \approx 7.3 \times 10^{-4}$. The hypothesis achieving consistency across all independent detectors is exponentially more likely to be correct than alternatives. $\square$
\end{proof}

\subsection{Pixel Demon Grid for Imaging}

An image is represented as a grid of pixel Maxwell demons:

\begin{definition}[Pixel Demon Grid]
\label{def:pixel_demon_grid}
A pixel demon grid of dimensions $(N_x, N_y)$ over physical extent $(L_x, L_y)$ consists of PMDs at positions:
\begin{equation}
\mathbf{r}_{i,j} = \left(\frac{i L_x}{N_x}, \frac{j L_y}{N_y}, 0\right), \quad i \in [0, N_x-1], \, j \in [0, N_y-1]
\end{equation}
\end{definition}

Each pixel independently queries its local categorical state, producing an image as the $N_x \times N_y$ array of knowledge entropy values:
\begin{equation}
I[i,j] = S_k(\mathbf{r}_{i,j})
\end{equation}

The pixel demon grid enables parallel categorical queries across all image locations simultaneously, with computational complexity $\mathcal{O}(N_x \times N_y \times m)$ where $m \approx 5$ is the number of molecular species, achieving real-time performance for typical image resolutions.

\subsection{Computational Complexity Analysis}

\begin{theorem}[Pixel Demon Query Complexity]
\label{thm:pixel_demon_complexity}
A categorical state query at position $\mathbf{r}$ has computational complexity:
\begin{equation}
\mathcal{O}(|\mathcal{M}|) = \mathcal{O}(m)
\end{equation}
independent of total molecular count $N_{\text{total}}$.
\end{theorem}

\begin{proof}
The categorical state $\mathbf{S}(\mathbf{r})$ is computed from molecular demon lattice properties:
\begin{equation}
\mathbf{S} = F(n_1, f_1, \phi_1, \ldots, n_m, f_m, \phi_m)
\end{equation}

This requires $m$ aggregation operations (one per species), regardless of how many individual molecules of each species exist at position $\mathbf{r}$. The molecular demon $M_i$ pre-aggregates information from all type-$i$ molecules through:
\begin{align}
n_i &= \sum_{j \in \text{type}_i} 1 \\
\phi_i &= \arg\left(\sum_{j \in \text{type}_i} e^{i\phi_j}\right)
\end{align}

These aggregations are performed once during molecular demon lattice initialization. Subsequent queries access pre-computed aggregates in $\mathcal{O}(1)$ time per species, yielding total complexity $\mathcal{O}(m)$. For atmospheric conditions with $m = 5$ species and $N_{\text{total}} \sim 10^{25}$ molecules, the reduction from $\mathcal{O}(N_{\text{total}})$ to $\mathcal{O}(m)$ renders categorical queries computationally tractable. $\square$
\end{proof}

For a complete image grid of $N_x \times N_y$ pixels:
\begin{equation}
\text{Complexity}_{\text{grid}} = \mathcal{O}(N_x \times N_y \times m)
\end{equation}

For a 1024 $\times$ 1024 image with $m = 5$ species, this yields approximately $5 \times 10^6$ operations, achievable in milliseconds on contemporary hardware, enabling real-time categorical image representation.

