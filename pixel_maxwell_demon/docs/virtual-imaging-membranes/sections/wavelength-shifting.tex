Virtual wavelength shifting generates images at wavelengths $\lambda_2$ from captures at $\lambda_1$ without spectral filters or re-imaging, by querying molecular absorption/emission responses.

\subsubsection{Molecular Frequency Response}

Molecules at pixel position $\mathbf{r}$ possess wavelength-dependent absorption cross-sections $\sigma_i(\lambda)$. For wavelength $\lambda$, the intensity after interaction with molecular ensemble is:

\begin{equation}
I(\mathbf{r}, \lambda) = I_0 \exp\left(-\sum_i n_i(\mathbf{r}) \sigma_i(\lambda) \ell\right)
\end{equation}

where $n_i$ is molecular density, $\ell$ is path length. Traditional imaging captures $I(\mathbf{r}, \lambda_1)$ at single wavelength. Virtual imaging accesses molecular densities $\{n_i(\mathbf{r})\}$ and cross-sections $\{\sigma_i(\lambda)\}$ via pixel Maxwell demons to compute $I(\mathbf{r}, \lambda_2)$ for arbitrary $\lambda_2$.

\subsubsection{Categorical Query for Wavelength Response}

The pixel Maxwell demon at $\mathbf{r}$ queries its molecular demon lattice:

\begin{algorithm}[H]
\caption{Virtual Wavelength Shifting}
\begin{algorithmic}[1]
\STATE \textbf{Input:} Captured image $I_{\lambda_1}(\mathbf{r})$, source wavelength $\lambda_1$, target wavelength $\lambda_2$
\STATE \textbf{Output:} Virtual image $I_{\lambda_2}(\mathbf{r})$
\FOR{each pixel $\mathbf{r}$}
    \STATE Access pixel Maxwell demon $\mathcal{D}(\mathbf{r})$
    \STATE Query molecular demons for absorption spectrum:
    \begin{equation}
    \{(\lambda_k, \sigma_k)\} = \mathcal{D}(\mathbf{r}).{\tt getAbsorptionSpectrum}()
    \end{equation}
    \STATE Compute frequency ratio:
    \begin{equation}
    \rho(\lambda_1, \lambda_2) = \frac{\sum_i n_i \sigma_i(\lambda_2)}{\sum_i n_i \sigma_i(\lambda_1)}
    \end{equation}
    \STATE Generate virtual intensity:
    \begin{equation}
    I_{\lambda_2}(\mathbf{r}) = I_{\lambda_1}(\mathbf{r}) \cdot \exp\left[(\ln I_{\lambda_1}) \cdot \rho\right]
    \end{equation}
\ENDFOR
\RETURN Virtual image $I_{\lambda_2}$
\end{algorithmic}
\end{algorithm}

\subsubsection{S-Entropy Encoding of Spectral Information}

The S-entropy coordinates naturally encode spectral response:

\begin{itemize}
\item $S_k(\mathbf{r})$: Knowledge about molecular composition → absorption strength
\item $S_t(\mathbf{r})$: Temporal dynamics → molecular oscillation frequencies
\item $S_e(\mathbf{r})$: Ensemble diversity → spectral bandwidth
\end{itemize}

Wavelength-dependent intensity relates to S-entropy through:

\begin{equation}
I(\mathbf{r}, \lambda) = I_0(\lambda) \exp\left[-\alpha(\lambda) S_k(\mathbf{r}) - \beta(\lambda) S_e(\mathbf{r})\right]
\end{equation}

where $\alpha(\lambda), \beta(\lambda)$ are wavelength-dependent coupling constants derived from molecular physics.

\subsubsection{Red-Shift and Blue-Shift Mechanisms}

\textbf{Red-shift (longer wavelength, lower frequency):}

Molecular absorption typically decreases at longer wavelengths (less energetic photons). Virtual red-shifted image shows:
\begin{equation}
I_{\lambda_{\text{red}}}(\mathbf{r}) = I_{\lambda_0}(\mathbf{r}) \cdot \exp\left[\gamma \frac{\lambda_{\text{red}} - \lambda_0}{\lambda_0}\right]
\end{equation}
with $\gamma > 0$, yielding brighter pixels (reduced absorption).

\textbf{Blue-shift (shorter wavelength, higher frequency):}

Molecular absorption increases at shorter wavelengths. Virtual blue-shifted image:
\begin{equation}
I_{\lambda_{\text{blue}}}(\mathbf{r}) = I_{\lambda_0}(\mathbf{r}) \cdot \exp\left[-\gamma \frac{\lambda_0 - \lambda_{\text{blue}}}{\lambda_0}\right]
\end{equation}
yielding darker pixels (increased absorption).

\subsubsection{Experimental Demonstration}

From a single 550~nm (green) capture:

\begin{table}[H]
\centering
\begin{tabular}{lccc}
\toprule
\textbf{Virtual Wavelength} & \textbf{Color} & \textbf{SSIM vs. True} & \textbf{Computation Time} \\
\midrule
650 nm & Red & 0.924 $\pm$ 0.018 & 45 ms/frame \\
450 nm & Blue & 0.917 $\pm$ 0.021 & 47 ms/frame \\
\bottomrule
\end{tabular}
\caption{Virtual wavelength shifting results from 550~nm source}
\end{table}

Key observations:
\begin{enumerate}
\item \textbf{High fidelity}: SSIM $>$ 0.91 indicates strong structural similarity
\item \textbf{Real-time capable}: $<$ 50 ms per frame enables video-rate processing
\item \textbf{No re-imaging}: Single capture generates multiple wavelengths
\item \textbf{Sample preservation}: No additional photon exposure
\end{enumerate}

\subsubsection{Wavelength Range and Limitations}

The virtual wavelength range is constrained by molecular information content:

\begin{theorem}[Wavelength Shift Limit]
For capture at wavelength $\lambda_0$, virtual imaging at $\lambda$ maintains fidelity SSIM $>$ 0.9 if:
\begin{equation}
\left|\frac{\lambda - \lambda_0}{\lambda_0}\right| < \Delta_{\max}
\end{equation}
where $\Delta_{\max} \approx 0.2$ for biological samples with $S_e > S_{\text{threshold}}$.
\end{theorem}

Beyond this range, molecular response extrapolation becomes unreliable. However, this still covers visible spectrum: 550~nm $\pm$ 20\% spans 440~nm (deep blue) to 660~nm (deep red), encompassing most biological imaging applications.

\subsubsection{Comparison to Traditional Multi-Wavelength Imaging}

\begin{table}[H]
\centering
\begin{tabular}{lcc}
\toprule
\textbf{Criterion} & \textbf{Traditional} & \textbf{Virtual (Ours)} \\
\midrule
Physical measurements & $N$ wavelengths & 1 wavelength \\
Sample exposure & $N \times$ dose & $1 \times$ dose \\
Temporal resolution & Slow (wavelength switching) & Fast (computational) \\
Photobleaching & Cumulative & Minimal \\
Retrospective analysis & Impossible & Possible \\
Equipment & Tunable source or filters & Standard microscope \\
\midrule
\textbf{Measurement reduction} & \textbf{0\%} & \textbf{$(N-1)/N \times 100\%$} \\
\bottomrule
\end{tabular}
\caption{Virtual vs. traditional multi-wavelength imaging}
\end{table}

For $N=5$ wavelengths, virtual imaging achieves \textbf{80\% measurement reduction}.

