\documentclass[11pt,a4paper]{article}
\usepackage[utf8]{inputenc}
\usepackage[margin=0.75in]{geometry}
\usepackage{amsmath}
\usepackage{amsfonts}
\usepackage{enumitem}
\usepackage{hyperref}
\usepackage{titlesec}
\usepackage{graphicx}
\usepackage{wrapfig}
\usepackage{multicol}

\titleformat{\section}{\large\bfseries}{\thesection}{1em}{}
\titleformat{\subsection}{\normalsize\bfseries}{\thesubsection}{1em}{}

\pagestyle{empty}

\begin{document}

% Header with photo
\begin{minipage}[t]{0.65\textwidth}
{\LARGE \textbf{Kundai Farai Sachikonye}}\\[0.3cm]
Computational Biology and Biophysics\\
GitHub: github.com/fullscreen-triangle\\
Email: kundai.sachikonye@wzw.tum.de\\
Phone: \\
Address: Auf der Lichtung 19, 82971 Puchheim, Germany
\end{minipage}
\hfill
\begin{minipage}[t]{0.3\textwidth}
\vspace{-0.5cm}
% Professional photo placeholder - replace 'photo.jpg' with your actual photo file
\fbox{\parbox{4cm}{\centering 
\vspace{1.5cm}
\textbf{Professional}\\
\textbf{Photo}\\
\textbf{4cm x 5cm}\\
\vspace{1.5cm}
}}
% Uncomment the line below and add your photo file:
% \includegraphics[width=4cm,height=5cm]{photo.jpg}
\end{minipage}

\vspace{0.5cm}

\section{Education}

\subsection{Academic Background}
\begin{itemize}[leftmargin=*,itemsep=0.1em]
\item \textbf{PhD Computational Lipidomics} (2019-2021) - Technische Universität München (Incomplete)
\item \textbf{MSc Computational Biology} (2018-2019) - Universität Tübingen (Incomplete)
\item \textbf{MSc Pharmaceutical Biotechnology} (2014-2017) - HAW Hamburg
\item \textbf{BSc Biochemistry and Cell Biology} (2011-2014) - Jacobs University Bremen
\end{itemize}

\subsection{Relevant Coursework}
\begin{itemize}[leftmargin=*,itemsep=0.1em]
\item \textbf{Computational Methods}: Molecular dynamics simulations, quantum chemistry calculations
\item \textbf{Biophysics}: Membrane dynamics, protein folding, cellular energetics
\item \textbf{Analytical Chemistry}: Mass spectrometry, spectroscopic methods, data analysis
\item \textbf{Programming}: Scientific computing, algorithm development, data processing
\end{itemize}

\section{Work Experience}

\subsection{Software Development and Computational Research}

\textbf{Bitspark GmbH} (June 2021 - Present)\\
\textit{Full Stack Developer - Web Development and Federated Learning}
\begin{itemize}[leftmargin=*,itemsep=0.1em]
\item Development of extensible JavaScript platform for IoT and smart wearable technology integration
\item Real-time sensor data aggregation using serverless functions
\item Implementation of personalized, decentralized distributed client systems
\item Development of computational frameworks for biological system modeling
\item Created biomimetic computing architectures using quantum mechanical principles
\item Implemented molecular dynamics simulations for membrane processes
\end{itemize}

\textbf{Bayerisches Forschungs Institute} (December 2019 - June 2021)\\
\textit{Full Stack Developer - Computational Lipidomics}
\begin{itemize}[leftmargin=*,itemsep=0.1em]
\item Developed web platform for federated learning in mass spectrometry-based lipidomics
\item Java implementation for peak finding algorithms
\item PHP development for ion database annotation systems
\item Python feature engineering for spectroscopic data analysis
\item TypeScript front-end for data visualization and statistical routines
\end{itemize}

\textbf{Fraunhofer Institut IGB} (February 2019 - August 2019)\\
\textit{Full Stack Developer - Bioprocess Automation}
\begin{itemize}[leftmargin=*,itemsep=0.1em]
\item Online real-time process monitoring for industrial algae photo-bioreactors
\item Data acquisition and processing modules in C and Python
\item Dynamic and linear programming methods for production optimization
\item Therapeutic lipid production automation systems
\end{itemize}

\subsection{Previous Research Experience}
\begin{itemize}[leftmargin=*,itemsep=0.1em]
\item \textbf{Uniklinikum Eppendorf} (2017-2018): Automated proteomics/glycomics analysis tools
\item \textbf{Anspach AG} (2016): Liposomal encapsulation efficiency optimization
\item \textbf{Leibniz Lungenzentrum} (2013): Anti-TB drug efficacy measurement using LC-MS
\item \textbf{University Medical Center Groningen} (2013): Cervical cancer biomarker discovery
\item \textbf{University of Zimbabwe} (2012): Plant metabolite characterization pipeline
\end{itemize}

\section{Technical Skills}

\subsection{Programming Languages}
\begin{itemize}[leftmargin=*,itemsep=0.1em]
\item \textbf{Scientific Computing}: Python, Rust, Matlab, C/C++
\item \textbf{Web Development}: TypeScript, JavaScript, PHP, HTML5, CSS
\item \textbf{Database Systems}: PostgreSQL, MySQL, Neo4j, GraphQL
\item \textbf{Frameworks}: React, Angular, Django, Flask, Node.js
\end{itemize}

\subsection{Scientific Software and Methods}
\begin{itemize}[leftmargin=*,itemsep=0.1em]
\item \textbf{Molecular Modeling}: NAMD, GROMACS, VMD, molecular dynamics simulations
\item \textbf{Data Analysis}: SciPy, NumPy, Pandas, TensorFlow, PyTorch
\item \textbf{Mass Spectrometry}: LC-MS/MS data processing, peak finding algorithms
\item \textbf{Computational Biology}: Protein folding analysis, membrane dynamics
\item \textbf{Image Processing}: OpenCV, computer vision algorithms
\item \textbf{Statistical Analysis}: R, statistical modeling, data visualization
\end{itemize}

\subsection{Analytical Techniques}
\begin{itemize}[leftmargin=*,itemsep=0.1em]
\item \textbf{Mass Spectrometry}: LC-MS/MS, MALDI-TOF, proteomics, lipidomics
\item \textbf{Spectroscopy}: UV-Vis, fluorescence spectroscopy
\item \textbf{Microscopy}: Fluorescence microscopy, atomic force microscopy
\item \textbf{Cell Biology}: Cell culture, biochemical assays, protein analysis
\end{itemize}

\section{Research Interests and Projects}

\subsection{Computational Biology and Biophysics}
\begin{itemize}[leftmargin=*,itemsep=0.1em]
\item \textbf{Membrane Dynamics}: Computational modeling of biological membrane processes
\item \textbf{Molecular Systems}: Biomimetic computing architectures and quantum biological processes  
\item \textbf{Image Processing}: Computer vision algorithms for biological data analysis
\item \textbf{Thermodynamic Modeling}: Information processing in biological systems
\end{itemize}

\subsection{Software Development Projects}
\begin{itemize}[leftmargin=*,itemsep=0.1em]
\item \textbf{GitHub Portfolio}: Multiple computational biology and biophysics projects (github.com/fullscreen-triangle)
\item \textbf{Biological Simulation}: Molecular dynamics and membrane modeling frameworks
\item \textbf{Data Analysis Tools}: Mass spectrometry and spectroscopic data processing
\item \textbf{Web Platforms}: Scientific computing and visualization applications
\end{itemize}

\subsection{Personal Information}
\begin{itemize}[leftmargin=*,itemsep=0.1em]
\item \textbf{Date of Birth}: 18th January 1992
\item \textbf{Nationality}: Zimbabwean
\item \textbf{Languages}: English (Native), German (Conversational)
\item \textbf{Availability}: Immediate start possible
\end{itemize}

\section{Publications and Documentation}

\subsection{Technical Documentation}
\begin{itemize}[leftmargin=*,itemsep=0.1em]
\item \textbf{Software Documentation}: Comprehensive documentation for computational biology frameworks
\item \textbf{Algorithm Implementation}: Detailed implementation of molecular dynamics algorithms
\item \textbf{Data Processing Pipelines}: Documentation of mass spectrometry data analysis workflows
\item \textbf{Open Source Contributions}: Multiple repositories with documented computational methods
\end{itemize}

\section{References}

\subsection{Academic and Professional References}
\begin{itemize}[leftmargin=*,itemsep=0.1em]
\item \textbf{Prof. Dr. Hartmut Schlueter} - University Klinik Eppendorf, Clinical Chemistry\\
Email: hschluet@uke.de, Phone: +49 (0) 40 7410 - 58795
\item \textbf{Prof. Dr. Sebastian Springer} - Jacobs University Bremen\\
Email: s.springer@jacobs-university.de  
\item \textbf{Dr. Christopher Krisp} - Macquarie University and University Klinik Hamburg-Eppendorf\\
Email: c.krisp@uke.de
\item \textbf{Prof. Dr. Birger Anspach} - HAW Hamburg, Technical Biochemistry\\
Email: birger.anspach@haw-hamburg.de
\end{itemize}

\end{document}



